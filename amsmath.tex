\documentclass{article}
\usepackage{amsmath}
\usepackage[spanish]{babel}

\begin{document}
Hola a todos, amiguitos.
\[
\underbrace{
    \begin{matrix}
    1 & 2 & 3 \\
    4 & 5 & 6 \\
    7 & 8 & 9
    \end{matrix}
}_{\textrm{Ejemplo de \texttt{matrix}}}
\]

\[
\underbrace{
    \begin{pmatrix}
    1 & 2 & 3 \\
    4 & 5 & 6 \\
    7 & 8 & 9
    \end{pmatrix}
}_{\textrm{Ejemplo de \texttt{pmatrix}}}
\]

\[
\underbrace{
    \begin{bmatrix}
    1 & 2 & 3 \\
    4 & 5 & 6 \\
    7 & 8 & 9
    \end{bmatrix}
}_{\textrm{Ejemplo de \texttt{bmatrix}}}
\]
\[
\underbrace{
    \begin{Bmatrix}
    1 & 2 & 3 \\
    4 & 5 & 6 \\
    7 & 8 & 9
    \end{Bmatrix}
}_{\textrm{Ejemplo de \texttt{Bmatrix}}}
\]
\[
\underbrace{
    \begin{vmatrix}
    1 & 2 & 3 \\
    4 & 5 & 6 \\
    7 & 8 & 9
    \end{vmatrix}
}_{\textrm{Ejemplo de \texttt{vmatrix}}}
\]
\[
\underbrace{
    \begin{Vmatrix}
    1 & 2 & 3 \\
    4 & 5 & 6 \\
    7 & 8 & 9
    \end{Vmatrix}
}_{\textrm{Ejemplo de \texttt{Vmatrix}}}
\]
\[
\overbrace{\boxed{1+1}}^{\textrm{Ejemplo de boxed}}
\]
Integrales más atractivas con límites encima y debajo en vez de a la derecha
\[
\int\limits_{0}^{-10}{x \: dx}
\]
\[
\iint\limits_{0}^{-10}{x \: dx}
\]
\[
\iiint\limits_{0}^{-10}{x \: dx}
\]
\end{document}
