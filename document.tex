%%%%%%%%%%%%%%%%%%%%%%%%%%%%%PREÁMBULO DEL DOCUMENTO%%%%%%%%%%%%%%%%%%%%%%%%%%%
% En esta sección se configura el documento, pero no se incluye contenido

% Esto es la clase del documento, que define las instrucciones principales
% que condicionarán el documento generado. Generalmente cambiarla después de 
% añadir contenido produce numerosos errores.
\documentclass[a4paper]{article}


%%%%%%%%%%%%%%%%%%%SECCIÓN INCLUSIÓN DE PAQUETES %%%%%%%%%%%%%%%%%%%%%%%%%%%%%%
% Este paquete permite cambiar la fuente. Hay muchas, la lista se puede
% encontrar en https://tug.org/FontCatalogue/, las que no indican nada después
% del nombre son las que se pueden usar como está aquí puesto.
\usepackage[T1]{fontenc}
% Comentar esta línea haría que se usara la fuente predeterminada.
%\usepackage{newcent}
\usepackage{titlesec}

% Carga el paquete babel, las opciones son:
    % spanish: indica que usaremos configuración para escribir en español
    % es-tabla: traduce la palabra table como tabla, no como cuadro
    % es-lcroman: permite la numeración romana en minúscula.
\usepackage[spanish,es-tabla]{babel}
% Paquete que permite escribir el símbolo del euro, se usa de dos maneras.
    % En el documento, la orden \euro escribirá el símbolo
    % Si se escribe \EUR{<cifra>} el paquete formará un bloque con el
    % número y el símbolo.
    % El símbolo del euro que pone es el «oficial» es decir, no se adapta
    % a la fuente que se use.
\usepackage{eurosym} 
% Paquete para que las imágenes tengan un tamaño relativo al del párrafo
\usepackage{graphicx}
% nos permite insertar las figuras con el especificador H, que hace que se
    % inserte en el lugar del document donde se ha escrito su código.
\usepackage{float} 
% Paquete que habilita las referencias cruzadas.
\usepackage{hyperref}
% Permite poner pies y encabezados de página personalizados
\usepackage{fancyhdr}
 %permite referenciar la última página (para incluir páginas totales)
\usepackage{lastpage}
% Paquete BibTex para referencias bibliográficas.
\usepackage[backend=biber, style=apa, citestyle=apa]{biblatex}
% Añadimos un archivo de fuentes bibliográficas.
\addbibresource{bibliography/sources.bib}
% Paquete para cambiar la configuración de página.
\usepackage{geometry}
%Cambiamos los márgenes de página
\geometry{
    a4paper, % tamaño del papel
    left=30mm, % márgenes laterales
    bottom=35mm, %margen inferior
    headheight=25mm    %margen superior
}
% Permite el uso de más caracteres, aunque algunos especiales siguen sin
% estar disponibles.
\usepackage[utf8]{inputenc}

%Permite utilizar la fuente de trazo doble en ecuaciones.
\usepackage{amsfonts}
%Permite figuras de figuras
\usepackage{subcaption}
%coloreado syntaxis
\usepackage{listings}
% mejores colores
\usepackage[dvipsnames,table]{xcolor}
    \definecolor{gray}{rgb}{0.5,0.5,0.5}
    \definecolor{lightYellow}{RGB}{251, 255, 212}
    \definecolor{lightBlue}{RGB}{56,184,255}
% Permite tachar ecuaciones
\usepackage{cancel}
% Permite celdas de varias filas en tablas
\usepackage{multirow}
\usepackage{array}
% Permite rotar texto
\usepackage{rotating}
%Permite definir nuevos entornos flotantes (como las figures
\usepackage{newfloat}
% Permite añadir captions a cosas personalizadas
\usepackage{caption}

\DeclareFloatingEnvironment[fileext=ecc,
                            placement={H},
                            name=Ecuación,
                            listname={Índice de ecuaciones}]
                            {ecuacion}
\captionsetup[ecuacion]{labelfont=normal}

\usepackage{multicol}

\usepackage{pdfpages}

%\usepackage{amsopn}
%%%%%%%%%%%%%%%%%%%%%%%%%%%%%% FIN SECCIÓN %%%%%%%%%%%%%%%%%%%%%%%%%%%%%%%%%%%%%
\lstset{
    basicstyle=\ttfamily,
    numberstyle=\ttfamily,
    frame=rlbt, % draw a frame at the top and bottom of the code block
    tabsize=4, % tab space width
    showstringspaces=false, % don't mark spaces in strings
    numbers=left, % display line numbers on the left
    commentstyle=\color{gray}, % comment color
    keywordstyle=\color{blue}, % keyword color
    stringstyle=\color{red}, % string color
    backgroundcolor=\color{lightYellow},
    captionpos=b,
    breaklines=true,
    literate=
            {Á}{{\'{A}}}1
            {É}{{\'{E}}}1
            {Í}{{\'{I}}}1
            {Ó}{{\'{O}}}1
            {Ú}{{\'{U}}}1
            {á}{{\'{a}}}1
            {é}{{\'{e}}}1
            {í}{{\'{i}}}1
            {ó}{{\'{o}}}1
            {ú}{{\'{u}}}1
            {ñ}{{\~{n}}}1
            {Ñ}{{\~{N}}}1
            {ü}{{\"{u}}}1
            {Ü}{{\"{U}}}1
            {¡}{{!`}}1
            {¿}{{?`}}1
}
\renewcommand{\lstlistingname}{Programa}
\renewcommand{\lstlistlistingname}{Índice de \MakeLowercase{\lstlistingname}s}
\newcommand{\mod}{\mathop{\mathrm{m\acute{o}d}}}
% Vamos a permitir más niveles de título, hasta 5.
\setcounter{secnumdepth}{5}
% Esto hace que la tabla de contenidos muestre hasta el nivel 3.
    % Si se quiere que la tabla de contenido muestre todos los niveles, sólo
    % ha de cambiarse el 3 por un 5
\setcounter{tocdepth}{3}
% Aquí estamos redefiniendo el comando paragraph, será nuestro título de nivel
    % 4. Esta es una de las cosas que no tengo nada claro cómo funciona, 
    % pero funciona, así que si quieres usar títulos de esta jerarquía, 
    % déjalo así. 
\titleformat{\paragraph}{\bfseries}{\theparagraph .}{1em}{}
\titlespacing*{\paragraph}{0pt}{3.25ex plus 1ex minus .2ex}{1.5ex plus .2ex}

% El subparagraph será el título de nivel 5
\titleformat{\subparagraph}{\bf}{\thesubparagraph .}{1em}{}
\titlespacing*{\subparagraph}{0pt}{3.25ex plus 1ex minus .2ex}{1.5ex plus .2ex}


% Esta línea nos permite indicar un directorio donde estamos guardando
% nuestras imágenes. Más información más adelante.
\graphicspath{{img/}}


% Esto hace que los enlaces a webs (\href{url}{texto} salgan en azul.
    % También impide que salga un repugnante cuadrado rojo alrededor de cada
    % referencia cruzada que incluyas en el documento.
\hypersetup{
    colorlinks=true,
    linkcolor=black,
    filecolor=magenta,      
    urlcolor=blue,
    citecolor=black
}

% Permite que en las ecuaciones escribas un punto y salga un punto (no lo
% interprete como un decimal en español, es decir, una coma)
\decimalpoint


%%%%%%%%%%%%%%%%%SECCIÓN VARIABLES %%%%%%%%%%%%%%%%%%%%%%%%%%%%%%%%%%%%%%%%%%%%
% Podemos definir variables para usar a lo largo del texto
% Así podemos cambiarlas aquí sin tener que repetir texto.
\def \autor{Francisco Rodríguez Melgar}
\def \titulo{Ejemplo de documento en \LaTeX{}}
\def \organizacion{Universidad de Ejemplo}
%%%%%%%%%FIN SECCIÓN%%%%%%%%%%%%%%%%%%%%%%%%%%%%%%%%%%%%%%%%%%%%%%%%%%%%%%%%%%%

% Esta orden especifica que cuando se cree la portada se use este título.
% Además, con Huge utilizamos la macro de tipo de letra más grande que hay 
% disponible.
\title{\textbf{\Huge{\titulo}}}
% Aquí lo mismo, pero con el autor
% Algunos especificadores de tamaño son: tiny, small, large, Large, LARGE, huge
    % y Huge
\author{\LARGE{\autor}\\ \\ \Large{\organizacion}}
%%%%%%%%%%%%%%%%%%%%%%%%%%%%FIN DEL PREÁMBULO%%%%%%%%%%%%%%%%%%%%%%%%%%%%%%%%%%


%%%%%%%%%%%%%%%%%%%%%%%%%%%%%%INICIO DEL DOCUMENTO%%%%%%%%%%%%%%%%%%%%%%%%%%%%%
% La totalidad del texto que se renderizará en PDF en tu documento debe estar
% entre begin document y end document.
\begin{document}
% Elimina la numeración de página hasta que se diga lo contrario.
\pagenumbering{gobble}
% Pone el título, el autor y la fecha. (la fecha se detecta automáticamente)
\maketitle

% Aquí creamos una figura para poner el logo de algo (Ahora hay un placeholder) 
% más en cuanto a figuras más adelante.
\begin{figure}[H]
    % Lo centramos
    \center
    \includegraphics[width=.5\linewidth]{escudo}
\end{figure}
\newpage
% En esta página puedes poner agradecimientos o prefacios que no tendrán pie ni
% encabezado de página y que no afectarán a la numeración, si no quieres poner
% nada, elimina esta línea y uno de los newpages
\begin{center}
\textbf{[Esta página ha sido dejada en blanco a propósito por el editor]}
\end{center}
\newpage
%Vamos a poner agradecimientos. El bloque flushright alinea el texto a la 
    % derecha.
\begin{flushright}
\textit{Esto es un placeholder, muchas gracias, placeholder.}
\end{flushright}
\newpage % salto de página

%%%%%%%%%%%%%%%%% CABECERA Y PIE DE PÁGINA SECCIÓN ÍNDICE%%%%%%%%%%%%%%%%%%%%%
\pagestyle{fancy}
\fancyhf{}
\fancyhead[C]{
\Large\titulo\normalsize\\
\rule[1mm]{0.3\hsize}{.5pt}\\
ÍNDICES
}
\fancyhead[R]{\includegraphics[height=2cm]{escudo}}
% Incluye el texto "pág. X de Y" en el pie de pág. a la derecha.
    % estoy referenciando una etiqueta que añadí, se verá luego.
\rfoot{pág. \textsc{\thepage{}} de \textsc{\pageref{startSectionContent}}} 
\fancyfoot[L]{\today}
% Incluye una línea en el pie de página.
\renewcommand{\footrulewidth}{0.5pt}
%decimos que se numeren las páginas en romano hasta nueva orden
    % Esto nos permite numerar las páginas de índices en números romanos
\pagenumbering{roman} 
%%%%%%%%%%%%%%%%%                 FIN                      %%%%%%%%%%%%%%%%%%%
\tableofcontents %crea el índice el índice empieza siempre en página nueva

% se pueden añadir saltos de página extra con \newpage
\newpage

% cambia el nombre del índice de ilustraciones
%\renewcommand{\listfigurename}{Índice de ilustraciones}
%inserta el índice de ilustraciones
\listoffigures 
\newpage

%cambia el nombre del índice de tablas
%\renewcommand{\listtablename}{Índice de tablas}
%inserta el índice de tablas
\listoftables
\newpage

\lstlistoflistings
\newpage
\listofecuacions
% Esta orden (label) inserta etiquetas invisibles en el texto, al ponerla justo
    % antes del salto de página del último índice de la sección, me permite
    % referenciar este punto, así es como consigo que salga "Pág. i de iii".
\label{startSectionContent}
\newpage
% A partir de aquí numeración normal
\pagenumbering{arabic}
%%%%%%%%%%%%%%%%% CABECERA Y PIE DE PÁGINA SECCIÓN CUERPO%%%%%%%%%%%%%%%%%%%%%%
\pagestyle{fancy}
\fancyhf{}
\fancyhead[C]{
\Large\titulo\normalsize\\
\rule[1mm]{0.3\hsize}{.5pt}\\
\leftmark
}
\fancyhead[R]{\includegraphics[height=2cm]{escudo}}
% El único cambio es que aquí referencio la última página del documento.
\fancyfoot[R]{pág. \thepage{} de \pageref{LastPage}}
\fancyfoot[L]{\today}
\renewcommand{\footrulewidth}{0.5pt}
%%%%%%%%%%%%%%%%%%%%%%%%%%%%%%%%%FIN SECCIÓN%%%%%%%%%%%%%%%%%%%%%%%%%%%%%%%%%%%
% La sección es el primer nivel de título que se permite en esta clase de 
    % documento. Introduce un título de primer nivel
\section{Presentación}\label{sec:presentacion}
Lo primero que debemos saber es qué es \LaTeX{}. \LaTeX{} es una herramienta
que nos permite crear documentos para mostrar, generalmente PDF
(\textit{Portable Document Format}), a través de la escritura de archivos de
texto plano. De una manera similar a como se escribe el código para crear
programas en lenguajes de programación como C, C++ o Java. Para crear los
archivos de texto nos valdrá cualquier editor, y para que se generen los PDF
usaremos un <<motor>>. Que no es más que un programa que leerá las órdenes
que hayamos incluido en el archivo de texto plano y generará el PDF a partir de
él.

Aunque en este documento no voy a hablar de las características históricas o de
lo que es el \textit{typesetting}. Sí voy a explicar someramente qué ventajas
tiene, por qué es una alternativa a un editor de texto <<normal>> y qué se puede
hacer con él. Además, comentaré en qué entornos es más usado.

\LaTeX{} es un sistema de creación de documentos donde el contenido del
documento y su formato están separados, es decir, no vemos el formato cuando
estamos editando el contenido del documento. Esto es muy distinto a lo que la 
mayoría de personas entiende por editar texto, en la mayoría de editores
modernos, el usuario ve lo que está editando con el formato que le está
añadiendo. Este tipo de editores de texto se llaman WYSIWYG, por las frase en
inglés, \textit{What You See Is What You Get}, es decir: lo que ves, es lo que
consigues. La principal desventaja de esto es que las opciones de maquetación
se <<ponen en el camino>> de nuestro contenido, impidiéndonos concentrar
nuestros esfuerzos en él. En \LaTeX{} esto no pasa.

Otra de las ventajas de \LaTeX{} es que, al utilizar texto plano la mayoría del
tiempo, en lugar de un documento completo, se puede editar en dispositivos con
pocos recursos, sólo necesitaremos más recursos en el proceso de visualizado de
nuestros PDF, pero no en la edición, y la compilación podrá llevar más tiempo,
pero el motor no corre el riesgo de cerrarse inesperadamente por falta de
recursos. Y, si lo hiciera, no perderíamos contenido, ni trabajo. Como nota
añadida, al utilizar ficheros de texto plano, nos permite utilizar en nuestro
documentos herramientas de control de versiones, como Git, si eres tecnológo,
especialmente informático, probablemente estés familiarizado con ellas, y las
uses diariamente.

Con \LaTeX{}, además, podemos crear muchos tipos de documentos PDF: libros de
texto, libros de ficción, presentaciones de transparencias (vulgarmente
llamadas \textit{powerpoints}), artículos científicos, tesis o tesinas de
universidad... En este tutorial intentaré dar datos para componer una variedad
de documentos distintos, en general explicaré cómo se hacen las cosas y cada
uno podrá generar el documento con las opciones que desee.

Esta versatilidad que hace \LaTeX{} sea muy utilizado en entornos
científico-técnicos. Muchos artículos científicos, conocidos como
\textit{papers}, son escritos con él. Si compilas este documento y lo 
observas, y alguna vez te has preguntado por qué tantos artículos <<se ven
igual>>, te darás cuenta de que es, en efecto, porque están escritos con
\LaTeX{}.

Este documento pretende ser un ejemplo de varias tareas que se suelen realizar
en \LaTeX{}. Se recomienda compilarlo y ver el PDF y el código a la
vez para ver los resultados. En los comentarios del código (en secciones 
posteriores explico lo que son los comentarios, si no lo sabes aún) explico lo 
que se va haciendo y qué órdenes son necesarias para crear lo que se ve en el
PDF. Además, incluyo un \textit{script} de compilación mínimo que debería
compilar sin problemas. Los prerrequisitos para utilizar la compilación
automática son:

\begin{itemize}
    \item Sistema UNIX like (GNU/Linux o Mac OS).
    \item Editor de ficheros de texto plano en el equipo (notepad, Notepad++,
        Sublime, Vim, Nano...)
    \item Programa pdflatex instalado en el equipo
    \item Utilidad make instalada en el equipo.
\end{itemize}

En caso de tener un sistema operativo Windows, deben seguirse las instrucciones
de instalación que se detallan en la siguiente sección.
\subsection{Instalación}
La instalación de \LaTeX{} es sencilla tanto en sistemas operativos GNU/Linux
como en sistemas operativos Windows.
\subsubsection{Instalación en GNU/Linux}\label{instGNULinux}
En la mayoría de instalaciones de escritorio de Linux se incluye un editor de
texto plano, ya sea éste Nano, Emacs, Vim u otros. Esos editores son válidos
para crear (o abrir) archivos \texttt{.tex}. Para instalar \LaTeX{} se puede
hacer en la mayoría de distribuciones mediante el gestor de paquetes, por
ejemplo, en las distribuciones derivadas de Debian (Ubuntu, Kubuntu, Xubuntu,
Linux Mint...) se puede instalar con la siguiente orden:
\begin{verbatim}
sudo apt install texlive-base \
texlive-latex-recommended \
texlive-latex-extra \
texlive-full
\end{verbatim} 
En otras distribuciones, 
por favor, consulta la documentación específica de cómo instalar paquetes de
\textit{software}.

Una vez instalados estos paquetes, se puede compilar con el comando
\texttt{pdflatex}, este comando debería ejecutarse sobre un archivo .tex.
(ver \textbf{\ref{ArchivosTex}}). Este tutorial se puede compilar
con el comando \texttt{make}. Para ello se debe tener instalada la utilidad 
\texttt{make}. Para hacerlo, de nuevo, en distribuciones derivadas de
Debian, se usa este comando:
\begin{verbatim}
sudo apt install make
\end{verbatim}

Si quieres preparar un entorno para compilar archivos (en Linux) que hayas
escrito tú, simplemente crea un directorio en un sitio sencillo (por ejemplo tu
carpeta personal) y, dentro del mismo, un archivo .tex. Por ejemplo, con estos
comandos de terminal se puede hacer:
\begin{verbatim}
mkdir ~/holaMundoLatex
touch ~/holaMundoLatex/holaMundo.tex
cd ~/holaMundoLatex
\end{verbatim}

Con estos comandos habrás creado un directorio en tu carpeta personal llamado
\texttt{holaMundoLatex}, un documento .tex dentro llamado \texttt{holaMundo.tex}
y te habrás situado en él para trabajar. En las páginas siguientes podrás ver
cómo crear el primer documento. Recuerda que el fichero de este tutorial dispone
de compilación con \texttt{make}, los que tú crees no.

\subsubsection{Instalación en Microsoft Windows}
En sistemas Windows, la instalación se realiza como otras herramientas de
\textit{software}, se descarga un \texttt{.exe} y se siguen las instrucciones
del instalador. En nuestro caso, debemos instalar dos programas:
\begin{itemize}
    \item MikTex, el <<motor>> de \LaTeX{}, que nos permitiría compilar,
pero sin ofrecernos asistencia a la edición.
Este programa se puede descargar desde \href{https://miktex.org/download}{aquí}. 
    \item TexStudio, el editor de documentos \LaTeX{}, que nos permite compilar
sin tener que usar la línea de comandos de Windows. Y nos ofrece
asistencia a la edición y funcionalidades como corrector ortográfico.
Se puede descargar desde \href{https://www.texstudio.org/}{aquí}.
\end{itemize}

Durante el proceso de
instalación de MikTex se nos hará decidir la ruta de instalación y si permitimos
que el programa descargue de Internet los paquetes necesarios. Yo
recomiendo permitirle hacerlo. Para ello se deben dejar las opciones como
se ve en la Figura \ref{fig:miktex}: \nameref{fig:miktex}.
\begin{figure}[H]
    \center
    \includegraphics[width=0.75\hsize]{miktexInstall}
    \caption{Opciones de instalación de MikTex}
    \label{fig:miktex}
\end{figure}

Cuando se tengan los dos programas instalados, deberá abrirse el fichero
\texttt{document.tex} que se encuentra en esta carpeta
con el segundo programa (TexStudio). Una vez hecho hecho, debe pulsarse la tecla
F5, que nos compilará el documento, generará el PDF y mostrará el resultado a la
derecha como se puede ver en la Figura \ref{fig:texstudio}:
\nameref{fig:texstudio}. Como este ejemplo incluye referencias bibliográficas,
no se van a crear bien, porque hacen falta pasos adicionales para eso, pero
se generará el PDF, y en secciones posteriores se explica lo que hay que hacer.

\begin{figure}[H]
    \center
    \includegraphics[width=1.0\hsize]{TexStudio}
    \caption{Ventana principal de TexStudio}
    \label{fig:texstudio}
\end{figure}

Este documento está diseñado para ser una guía básica sobre Latex, pero para
poder generarse, se usarán comandos en el mismo documento antes de explicarlos.
Si se sabe algo sobre \LaTeX{} es recomendable leer el código primero, si no, 
mejor leer primero el PDF compilado.

\newpage\section{Comandos básicos de \LaTeX{}}
En esta sección veremos los comandos más elementales de \LaTeX{}, cómo insertar
formatos simples, insertar secciones... Y, sobre todo, al principio explicaré
cómo crear tu primer documento con esta herramienta.

\subsection{Archivos .tex} \label{ArchivosTex}
Un archivo .tex es donde escribimos el código \LaTeX{}, es decir, la sucesión
de contenido y comandos que generará un documento. El
programa que genera el documento (generalmente PDF) a partir del código LaTeX
se llama motor o renderizador. Hay varios, con diferencias técnicas. Esta guía
y los comandos que se listan están desarrollados para el motor
\texttt{pdflatex}.

\subsubsection{Estructura de un archivo .tex}
Un documento en latex mínimo tiene estas líneas:
\begin{verbatim}
\documentclass{<clase del documento>}
\begin{document}
Algún texto.
\end{document}
\end{verbatim}

Para empezar: las palabras que van precedidas de una barra inclinada inversa
se llaman <<comandos>>, y son órdenes que se le dan al renderizador de \LaTeX{},
es decir, no se imprimirán en nuestro PDF (o no literalmente).
Debido a que los comandos se incluyen
con ese caracter especial, si quisiéramos escribir una barra inversa en el 
documento, deberíamos usar el comando \verb|\textbackslash|, 
seguido de un espacio. Si no, \LaTeX{} interpretaría que la siguiente palabra
es un comando. Además, los comandos reciben argumentos, que van entre llaves.
Como las llaves son otro carácter especial usado por el lenguaje, para 
incluirlas en nuestro texto, éstas deben ir precedidas de una barra inclinada
inversa, es decir, son otro comando: \verb|\{| produciría:
<<\{>> (sin las comillas).

La clase del documento será una palabra que definirá un estilo de documento
concreto que nos permita empezar a escribir sin tener que formatearlo todo
desde cero. Las más populares son: article, book y report; pero hay muchas más.
Si se quiere compilar el primer documento en LaTeX, se puede copiar ese código,
pero debe sustituirse \texttt{<clase del documento>} por \texttt{article}.
Este documento se ha escrito con la clase de documento article. Como las clases
de documento definen las instrucciones que se usarán después en el documento,
es muy difícil cambiarla una vez se han escrito y formateado una cuantas
páginas.

Todo lo que hay antes de la línea donde pone 
\verb|\begin{document}|, sin incluirla, se denomina el 
\textbf{preámbulo del documento}. Y lo que existe entre la línea
\verb|\begin{document}| y la línea
\verb|\end{document}| se considera el contenido del
documento. En el preámbulo se incluyen numerosas órdenes que afectarán a cómo
el documento se renderiza, una de las más importantes es la inclusión de
paquetes, esto se hace mediante la orden \verb|\usepackage|.
Estos paquetes permiten añadir muchas funcionalidades que son necesarias. De 
hecho, hoy en día cualquier documento moderno incluye muchos paquetes. Si
se observa el preámbulo de este documento, se verá que hay muchos.

\textbf{Nota sobre paquetes:} para realizar muchas de las tareas que se explican
aquí se han incluido paquetes, para saber qué paquete es necesario para cada
cosa, acudde a los comentarios del código.

Ahora que ya hemos visto el contenido mínimo de un archivo .tex, puedes abrir
aquél que creaste en el punto \ref{instGNULinux} y copiar los ejemplos
que estamos viendo en él, para compilarlo en Linux y crear un PDF, puedes
ejecutar (recuerda situarte en el directorio donde creaste el archivo):
\begin{verbatim}
pdflatex holaMundo.tex
\end{verbatim}
En Windows, puedes abrir el archivo .tex con TexStudio y compilarlo con F5.

Además, por su herencia o similitud con otros lenguages formales,
de programación y de marcado, \LaTeX{} tiene la posibilidad de incluir 
comentarios en el código. Esto es texto que el renderizador no lee. En \LaTeX{}
dichos comentarios se ponen con el signo de porcentaje <<\%>>. Cualquier
texto en un archivo \texttt{.tex} que vaya en la misma línea que un signo de
procentaje y después de él no se renderizará. \textbf{Esto incluye el salto
de lína del final}. De nuevo, si se quiere incluir un signo de porcentaje
en el texto, debemos incluir antes de él una barra inclinada inversa, es decir:
\verb|\%| generaría el signo de porcentaje.
Veamos un ejemplo:
\begin{verbatim}
\documentclass{article}
\begin{document} % Aquí puedo escribir lo que quiera; no va a salir
Algún texto de un párrafo.
% Este exto no va a salir
% Este tampoco.
Otro texto de otro párrafo.
\end{document}
\end{verbatim}

Además de que no se van a renderizar los textos de los comentarios, estos
dos párrafos son uno, porque se eliminan las líneas enteras, así que habría
que poner una línea en blanco entre el último comentario y el segundo
párrafo.

Al ser los documentos \texttt{.tex} documentos de texto plano, podríamos
incluir potencialmente los caracteres que quisiéramos, pero, sin paquetes 
adicionales (ver sección de inclusión de paquetes en el código de este
documento) no se pueden incluir algunos símbolos. Así que hay que tener cuidado
con los símbolos especiales que se usan, pueden renderizarse mal o no permitir
que el PDF se genere. Si se usan los símbolos del idioma español: tildes, la
eñe, diéresis en la u y demás, no debería haber problemas.
Ej.: áéíóú, ÁÉÍÓÚ, üÜ, ñÑ.

\subsection{Escritura de texto en \LaTeX{}} \label{subsec:write_text}
Como se ha visto en la sección anterior, dentro del contenido del documento es
donde se puede incluir texto. Ahora bien, si has experimentado con este archivo 
\texttt{.tex}.
que has creado con el código anterior, quizás hayas notado que los saltos de
línea no se trasladan a párrafos distintos.
En Latex, un párrafo es una línea de texto o varias líneas consecutivas, que se
renderizan siempre en el mismo párrafo salvo que haya una línea en blanco entre
ellas. Por ejemplo, vamos a poner un punto y aparte. Es decir:
\begin{verbatim}
\documentclass{article}
\begin{document}
Algún texto de un párrafo.

Otro texto de otro párrafo.
\end{document}
\end{verbatim}

Además, \LaTeX{} permite utlizar varios
comandos para manipular la presentación del texto.
\begin{itemize}
    \item \textbf{emph:} Pone el texto en énfasis.
    \item \textbf{textit:} Pone el texto en itálica (o cursiva).
        \textit{ejemplo}
    \item \textbf{textbf:} Pone el texto en negrita. \textbf{ejemplo}
    \item \textbf{texttt:} Utiliza fuente monoespaciada (como la de las máquinas
        de escribir). \texttt{ejemplo}
    \item \textbf{textsc:} Pone el texto en versalitas (letras del mismo tamaño
    que las minúsculas, pero que se escriben como mayúsculas). \textsc{ejemplo}.
\end{itemize}
Estos comandos deben ponerse en nuestros archivos \texttt{.tex}
con una barra inclinada hacia atrás delante, como todos los comandos de 
\LaTeX{} y seguidos de dos llaves,
dentro de las cuales incluiremos el texto que queramos modificar. Por ejemplo:
\begin{verbatim}
Este es el primer párrafo de esta sección,
en cualquier texto podemos incluir texto en
\textbf{negrita} o en \textit{cursiva}, incluso,
podríamos incluir texto en 
\textbf{\textit{negrita y cursiva}}. 
\end{verbatim}
sería el código en \LaTeX{} que renderizaría el párrafo que se ve a
continuación:


% párrafo con texto en distintos formatos, textit es itálita, textbf es 
% negrita, se pueden combinar. texttt es monoespaciada, la uso luego
% Como puedes ver, aunque hay saltos de línea, no son párrafos distintos.
% Para que sean párrafos distintos tiene que haber una línea en blanco
% entre ellos.
<<Este es el primer párrafo de esta sección, en cualquier texto
podemos incluir texto en
\textbf{negrita} o en \textit{cursiva}, incluso, podríamos incluir texto en 
\textbf{\textit{negrita y cursiva}}.>> (sin las comillas).

Si queremos escribir comandos u órdenes
de programación, podemos ponerlo en monoespaciado con
\verb|\texttt|, por ejemplo: \texttt{ls -la}.
Además, el comando de énfasis (\texttt{emph}) nos maneja el énfasis
automáticamente.
Por ejemplo, el código:
\begin{verbatim}
\emph{\emph{Toda} esta frase está en énfasis
menos la primera y la última \emph{palabra}}.
\end{verbatim}
Producirá el siguiente párrafo:

<<\emph{\emph{Toda} esta frase está en énfasis
menos la primera y la última \emph{palabra}}.>>
Esto es porque la orden emph, cuando se aplica a un texto enfatizado, elimina
la letra itálica, como es la norma.

Además, hay modificadores del tamaño del texto. Éstos no son absolutos, son
dependientes del tamaño de fuente predeterminada que usemos, que, si no se
indica nada, son 10 puntos.
\LaTeX{} nos ofrece 10 tamaños
de texto distintos, éstos son:

\begin{table}[H]
    \centering
    \begin{tabular}{|c|r|c|}
        \hline
        \textbf{Comando} & \textbf{Tamaño (relativo al normal)}&
                                                      \textbf{Ejemplo} \\\hline
        \verb|\tiny|&50 \% &\tiny{ejemplo}\normalsize \\\hline
        \verb|\scriptsize|&70 \% &\scriptsize{ejemplo}\normalsize \\\hline
        \verb|\footnotesize|&80 \% &\footnotesize{ejemplo}\normalsize \\\hline
        \verb|\small|&90 \% &\small{ejemplo}\normalsize \\\hline
        \verb|\normalsize|&100 \% &\normalsize{ejemplo}\normalsize \\\hline
        \verb|\large|&120 \% &\Large{ejemplo}\normalsize \\\hline
        \verb|\Large|&140 \% &\Large{ejemplo}\normalsize \\\hline
        \verb|\LARGE|&170 \% &\LARGE{ejemplo}\normalsize \\\hline
        \verb|\huge|&200 \% &\huge{ejemplo}\normalsize \\\hline
        \verb|\Huge|&250 \% &\Huge{ejemplo}\normalsize \\\hline
    \end{tabular}
    \caption{Comparación tamaños de letra}
    \label{tab:comLetSize}
\end{table}

Nótese que estos comandos cambian el tamaño de \textbf{todo el texto que vaya
tras ellos}, para hacer que el texto siguiente vuelva a ser de tamaño normal,
debe escribirse la orden \verb|\normalsize| al final del texto
cuyo tamaño queramos modificar.

Y, ya que se ha dicho que se pueden elegir tamaños distintos de letra para el
texto en general, veamos cómo. El catálogo de tamaños de letra que podemos
escoger se basa en nuestra clase de documento (\emph{documentclass}), 
las clases predeterminadas de \LaTeX{} como \texttt{article} sólo admiten
10 (opción predeterminada), 11 y 12 puntos. Esto se selecciona escribiendo
como opción el tamaño en la orden \texttt{documentclass}. Por ejemplo:
\begin{verbatim}
\documentclass[12 pt]{article}
\end{verbatim}
crearía un documento con un tamaño de letra predeterminado de 12 puntos, usando
la clase \textit{article}.

Además de texto, es natural que en un 
documento tengamos que poner listas, ya sean éstas numeradas o sin numerar.
A continuación vemos cómo se hace.
\begin{verbatim}
\begin{itemize}
\item Sector primario
\begin{itemize}
    \item Ganadería
        \begin{itemize}
            \item Porcina
            \item Bovina
            \item Avícola
        \end{itemize}
        \item Pesca
    \end{itemize}
    \item Sector Secundario
    \item Sector Servicios
\end{itemize}
\end{verbatim}

Ese código generaría la siguiente lista:
%%% Ejemplo de lista
% Este tipo de listas no son numeradas. Son listas de puntos (bullets)
\begin{itemize}
    \item Sector primario
    \begin{itemize}
        \item Ganadería
        \begin{itemize}
            \item Porcina
            \item Bovina
            \item Avícola
        \end{itemize}
        \item Pesca
    \end{itemize}
    \item Sector Secundario
    \item Sector Servicios
\end{itemize}

Para crear una lista con numeración se utilizará el mismo código que 
para una lista sin numeración, pero en vez de \texttt{itemize} pondremos
\texttt{enumerate}. El resultado sería:

\begin{enumerate}
    \item Sector primario
    \begin{enumerate}
        \item Ganadería
        \begin{enumerate}
            \item Porcina
            \item Bovina
            \item Avícola
        \end{enumerate}
        \item Pesca
    \end{enumerate}
    \item Sector Secundario
    \item Sector Servicios
\end{enumerate}
% Lista con enumeración (Nótese que entre llaves pone enumerate)
\subsection{Jerarquización del texto}
En el momento en que nuestro documento sea ligeramente más largo que una página
es probable que queramos añadir títulos de sección y de subsecciones.
Como se puede ver en el código, una sección se define fácilmente, con la
 orden \verb|\section{<título de la sección>}|, también existen
\verb|\subsection{<título de la sección>}| y
\verb|\subsubsection{<título de la sección>}|
A partir de ahí, si se requieren títulos de un nivel más
profundo, hay que usar otras órdenes. Esto depende de la clase de documento
escogida. En este caso, \textit{article}.
\subsubsection{Cómo definir niveles más profundos de la jerarquía}
Si quieres más información sobre ćomo hacer esto, puedes visitar
\href{https://ctan.org/pkg/titlesec}{este enlace}.
Pero es ciertamente algo complejo \textbf{en mi opinión}.
Sin embargo; como puede ser útil, he copiado unos comandos en el preámbulo
de este documento para permitir niveles mayores de título.
Comandos sacados de \href{https://tex.stackexchange.com/questions/60209/
how-to-add-an-extra-level-of-sections-with-headings-below-subsubsection}{aquí}.
De la respuesta de Gonzalo Medina.
Es una de las cosas de \LaTeX{} que no acabo de entender, así que simplemente me
limito a utilizar lo que otra gente ha implementado.

Voy a incluir a continuación dos títulos de profundidad cuatro y cinco 
respectivamente para que se vea la sintaxis en el código. Esto se ha conseguido
redefiniendo las órdenes \texttt{paragraph} y \texttt{subparagraph}. Por lo que
serán éstas las que se han de utilizar. 
\paragraph{Párrafo}
Este texto está bajo un epígrafe de nivel cuatro,
un párrafo, creado con la orden
\\ \verb"\paragraph{Párrafo}".

\subparagraph{Subpárrafo}
Este texto está bajo un epígrafe de nivel cinco, un párrafo, creado con la orden
\\ \verb"\subparagraph{Subpárrafo}". Este es el último nivel de jerarquía
admitido.
Cinco niveles de jerarquía parecen
más que suficientes para cualquier documento razonable.
\subsection{Ecuaciones en \LaTeX{}} \label{ecuaciones}
Uno de los puntos más fuertes de \LaTeX{} es la capacidad de incluir ecuaciones
matemáticas de manera simple. Si queremos una ecuación matemática en bloque
(separada del resto de los párrafos) pondríamos dos signos de dólar (\$), 
una línea nueva, la ecuación (que puede ocupar varias líneas) y otros dos signos
 de dólar. De nuevo, al ser el signo de dólar un símbolo especial de \LaTeX{},
si queremos incluirlo de manera literal, debe ir precedido de una barra
inclinada inversa: \verb|\$|. Por ejemplo:
\begin{verbatim}
$$
G\cdot\frac{m_1\cdot m_2}{d^2}
$$
\end{verbatim}
generaría la fórmula de la Ley de la Gravedad:
% Inicio bloque de ecuación.
$$
G\cdot\frac{m_1\cdot m_2}{d^2}
$$
Si en este párrafo quisiéramos poner una ecuación o números al estilo 
matemático, lo haremos con único signo de dólar a cada lado
y la escribiríamos
en la misma línea, por ejemplo, una ecuación cuadrática es: 
$ax^2+bx+c = d : a \ne 0$. Se ha escrito así:

\begin{verbatim}
[...]por ejemplo, una ecuación cuadrática es: 
$ax^2+bx+c = d : a \ne 0$. Se ha escrito así:[...]
\end{verbatim}
Algunas de las órdenes más comunes en matemáticas son:
\begin{enumerate}
\item Superíndices: \verb|$base^exponente$| produce:
$base^{exponente}$.
\item Subíndices:
\verb|$base_{sub\acute{\imath}ndice}$| produce: $base_{sub\acute{\imath}ndice}$.
    \begin{enumerate}
        \item La orden \verb|\imath| genera una i sin punto,
        existe un análogo para la j, \verb|\jmath|.
        \item la orden \verb|\acute| genera un acento agudo
        en la letra, sin el comando anteriormente mencionado, saldría encima
        del punto de la i.
    \end{enumerate}
\item Fracciones:  \verb|$\frac{numerador}{denominador}$.|
 produce: $\frac{numerador}{denominador}$.
\end{enumerate}


\subsubsection{Símbolos matemáticos avanzados}
Hay muchos símbolos y objetos matemáticos más avanzados que esos, aquí veremos
una selección de ellos. Probablemente añada más en el futuro. El primero
sería la matriz. Para crear una matriz se usaría:
\begin{verbatim}
\matrix{1 & 2 & 3 \cr 
        4 & 5 & 6 \cr
        7 & 8 & 9}
\end{verbatim}
Y el resultado sería:
$$
\matrix{1&2&3 \cr 
           4&5&6 \cr
           7&8&9}
$$
En \LaTeX{}, si queremos rodear cualquier parte de una ecuación con símbolos como
paŕentesis, corchetes y demás, lo tenemos que hacer con las órdenes 
\verb|\left| y \verb|\right| seguidas del
símbolo que queramos utilizar, por ejemplo si queremos una matriz rodeada
de paréntesis, ponemos:
\begin{verbatim}
$$
\left(\matrix{1&2\cr 3&4}\right)
$$
\end{verbatim}
Y generaría:

$$
\left(\matrix{1&2\cr 3&4}\right)
$$
Left y right son muy versátiles, porque permiten que utilizar muchos
delimitadores (tipos de paréntesis, por así decir) y, además, permiten que sean
distintos a un lado de la ecuación que a otro. Todo comando left debe cerrarse
con un comando right, pero los delimitadores especificados en ambos lados
pueden ser distintos, es decir: podríamos rodear una ecuación con un paréntesis
por la izquierda y un corchete por la derecha: $\left(n\atop k\right]$
Algunos delimitadores de left y right son:
\begin{table}[H]
\centering
\begin{tabular}{|c|c|}
\hline
\textbf{Delimitador}& \textbf{Ejemplo}\\\hline
\texttt{( )} &$\displaystyle\left(m\atop n\right)\vphantom{\matrix{\cr\cr}}$  \\\hline 
\texttt{[ ]} &$\displaystyle\left[m\atop n\right]\vphantom{\matrix{\cr\cr}}$  \\\hline 
\verb|{ }| &$\displaystyle\left\{m\atop n\right\}\vphantom{\matrix{\cr\cr}}$\\\hline 
\verb|\lceil \rceil| &
        $\displaystyle\left\lceil m\atop n\right\rceil\vphantom{\matrix{\cr\cr}}$ \\\hline 
\verb|\lfloor \rfloor|& 
        $\displaystyle\left\lfloor m\atop n\right\rfloor\vphantom{\matrix{\cr\cr}}$ \\\hline 
\verb!< >! &$\displaystyle\left< m\atop n\right>\vphantom{\matrix{\cr\cr}}$\\\hline 
\texttt{. .} &$\displaystyle\left. m\atop n\right.\vphantom{\matrix{\cr\cr}}$\\\hline 
\end{tabular}
\caption{Delimitadores para los comandos left y right}
\label{tab:delimForLeftRight}
\end{table}
Todos los delimitadores que se ven en la Tabla \ref{tab:delimForLeftRight}: 
\nameref{tab:delimForLeftRight} se pueden combinar entre ellos arbitrariamente,
pero, de nuevo, no puede haber un comando left sin su correspondiente right,
ni viceversa, aunque los delimitadores de ambos sean diferentes.
Hay que mencionar también los llamados <<operadores grandes>> que son aquéllos
como sumatorios, integrales... En general, se incluyen con un comando y 
con las órdenes de superíndice y subíndice después de él, se establecen
sus argumentos, por ejemplo:
$$
\sum_{i=1}^{n}{\frac{1}{n}}
$$

Se crearía con el ćodigo:

\begin{verbatim}
\sum_{i=1}^{n}{\frac{1}{n}}
\end{verbatim}

Las diferentes partes de esa ecuación serían:
\begin{itemize}
\item \verb|\sum|: comando que indica el sumatorio.
\item \verb|_{i=1}|: subíndice del sumatorio, indica lo que saldrá debajo de él.
\item \verb|^{n}|: Superíndice del sumatorio, indica lo que habrá encima.
\item \verb|{\frac{1}{n}}|: cuerpo del sumatorio, en este caso una fracción.
\end{itemize}

En la tabla \ref{tab:bigoperators} se presentan los operadores grandes con 
superíndices y subíndices. En algunos de estos operadores el superíndice no se
suele utilizar y sólo se muestra con motivo de presentación. En estos casos
éste será un asterisco: <<$*$>>.
\begin{table}[H]
\centering
\begin{tabular}{|c|c|c|}
\hline
\textbf{Comando} & \textbf{Ejemplo} & \textbf{Descripción} \\\hline
\verb|\sum| &$\displaystyle\sum_{i=1}^{n}{\frac{1}{n}}$ & Sumatorio\\\hline
\verb|\prod| & $\displaystyle\prod_{i=1}^{n}{\frac{1}{n}}$&Producto \\\hline
\verb|\coprod| & $\displaystyle\coprod_{j\in J}^{*}{\frac{1}{n}}$& Coproducto \\\hline
\verb|\int| &$\displaystyle \int_{0}^{\infty}{x\,dx} \vphantom{\matrix{1\cr1\cr1\cr}}$& Integral\\\hline
\verb|\oint| &$\displaystyle \oint_{0}^{\infty}{x\,dx}\vphantom{\matrix{1\cr1\cr1\cr}}$ & Integral de superficie\\\hline
\verb|\bigoplus| & $\displaystyle\bigoplus_{i=0}^{*}{s_i}$&Suma booleana exclusiva \\\hline
\verb|\bigotimes| & $\displaystyle\bigotimes_{i=0}^{n}{s_i}$&Producto tensorial \\\hline
\verb|\bigodot| & $\displaystyle\bigodot_{i=0}^{n}{s_i}$&Varios significados\\\hline
\verb|\bigcup| & $\displaystyle\bigcup_{i=1}^{n}{A_i}$& Unión de conjuntos\\\hline
\verb|\bigcap| & $\displaystyle\bigcap_{i=1}^{n}{A_i}$& Intersección de conjuntos\\\hline
\verb|\biguplus| & $\displaystyle\biguplus_{i=1}^{n}{A_i}$& Unión disjunta de conjuntos\\\hline
\verb|\bigsqcup| & $\displaystyle\bigsqcup_{i=1}^{n}{A_i}$&Variante de la unión\\\hline
\verb|\bigvee| & $\displaystyle\bigvee_{i=1}^{n}{p_i}$&  Disyunción lógica\\\hline
\verb|\bigwedge| & $\displaystyle\bigwedge_{i=1}^{n}{p_i}$&Conjunción lógica\\\hline
\verb|\lim| & $\displaystyle\lim_{x\to\infty}^{*}{\frac{1}{x}}\vphantom{\matrix{1\cr1\cr1\cr}}$&Límite\\\hline
\end{tabular}
\caption{Operadores matemáticos grandes}
\label{tab:bigoperators}
\end{table}

Además, como es lógico, \LaTeX{} incorpora las funciones matemáticas más usadas.
\begin{table}[H]
\centering
\begin{tabular}{|c|c|c|}
\hline
\textbf{Comando}&\textbf{Presentación}&\textbf{Función}\\\hline
\multicolumn{3}{|c|}{\textbf{Funciones básicas}}\\\hline
\verb|\sqrt[<grado>]{<argumento>}| &$\sqrt[3]{8}=2$ & Raíz enésima\\\hline
\multicolumn{3}{|c|}{\textbf{Funciones trigonométricas}}\\\hline
\verb|\sen| &$\sen{\alpha}$ & Seno\\\hline
\verb|\cos| &$\cos{\alpha}$& Coseno\\\hline
\verb|\tan| &$\tan{\alpha}$&Tangente \\\hline
\verb|\arcsen| &$\arcsen{x} $&Arcoseno \\\hline
\verb|\arccos| &$\arccos{x} $&Arcocoseno \\\hline
\verb|\arctan| &$\arctan{x} $&Arcotangente\\\hline
\verb|\sec| &$\sec{x}$&Secante \\\hline
\verb|\cosec| &$\cosec{x}$& Cosecante\\\hline
\verb|\cot| &$\cot{x}$&Cotangente \\\hline
\multicolumn{3}{|c|}{\textbf{Logaritmos}}\\\hline
\verb|\ln| &$\ln{x} $&Logaritmo neperiano\\\hline
\verb|\log_{<base>}{<argumento>}| &$\log_{3}{x}$&Logaritmo \\\hline
\end{tabular}
\caption{Algunas de las funcines incorporadas en \LaTeX{}}
\label{tab:foos}
\end{table}

Si se necesita una función que no esté aquí, por ejemplo, la función módulo,
se puede utilizar el comando \verb!\mathop!, que nos permite
hacer una función nueva, por ejemplo: \verb!\mathop{m\acute{o}d}{3}! sería
equivalentea $\mathop{m\acute{o}d}{3}$. Por desgracia, la nueva función sale
en cursiva, para evitar esto, debemos utilizar el comando \verb!\mathrm{}!, que
escribe dentro de una ecuación texto <<normal>>. Así que:
\begin{verbatim}
30 \equiv 3\mathop{\mathrm{m\acute{o}d}}{3}
\end{verbatim}
sería equivalente a:
$$
30 \equiv 3\mathop{\mathrm{m\acute{o}d}}{3}
$$

Como se puede ver, esta sintaxis se vuelve terriblemente engorrosa en cuanto
tenemos que usar nuestra flamante función nueva un par de veces en la misma
ecuación, pero no ha motivo de alarma, porque hay una manera de asignar un
conveniente comando a esta función. El comando \verb!\newcommand! nos permite
definir nuevos comandos en \LaTeX{}, de hecho, nos permite definir el comando
que queramos, pero nos centraremos en su uso para la casuística que aquí nos
ocupa. De este modo:
\begin{verbatim}
\newcommand{\mod}{\mathop{\mathrm{m\acute{o}d}}}
\end{verbatim}
sería la orden que definiría nuestro nuevo comando, si la ponemos, por ejemplo,
en el preámbulo de nuestro documento, el comando \verb!\mod! estaría disponible
en cualquier parte del mismo dentro de nuestras ecuaciones matemáticas. Así que
\begin{verbatim}
30 \equiv 6 \mod 3
\end{verbatim}
produciría, concisamente:
$$
30\equiv 6 \mod 3
$$

Hay, además, hay muchos operadores que son necesarios, algunos de ellos
son:
\begin{table}[H]
    \centering
    \begin{tabular}{|c|c|c|}
    \hline
    \textbf{Comando} & \textbf{Presentación} & \textbf{Descripción} \\ \hline
    \texttt{<} & $<$ &Es menor que\\ \hline
    \texttt{>} & $>$ &Es mayor que\\ \hline
    \texttt{=} & $=$ & Es igual a\\ \hline
    \verb|\le| & $\leq$ &Menor o igual que\\ \hline
    \verb|\ge| & $\geq$ &Mayor o igual que\\ \hline
    \verb|\subset| & $\subset$ &Es subconjunto de\\ \hline
    \verb|\supset| & $\supset$ &Es superconjunto de \\ \hline
    \verb|\subseteq| & $\subseteq$ &Es subconjunto o igual a  \\ \hline
    \verb|\supseteq| & $\supseteq$ &Es superconjunto o igual a \\ \hline
    \verb|\in| & $\in$ & Está en\\ \hline
    \verb|\prec| & $\prec$ & Precede a\\ \hline
    \verb|\preceq| & $\preceq$ & Precede o es igual a\\ \hline
    \verb|\succ| & $\succ$ & Antecede a\\ \hline
    \verb|\succeq| & $\succeq$ & Antecede o es igual a\\ \hline
    \verb|\equiv| & $\equiv$ & Equivale, coincide con\\ \hline
    \verb|\approx| & $\approx$ & Es, aproximadamente\\ \hline
    \verb|\cup| & $\cup$ & Unión de conjuntos\\ \hline
    \verb|\cap| & $\cap$ & Intersección de conjuntos\\ \hline
    \verb|\land| & $\land$ &Conjunción lógica\\ \hline
    \verb|\lor| & $\lor$ &Disyunción lógica\\ \hline
    \verb|\times| & $\times$ &Producto vectorial\\ \hline
    \verb|\cdot| & $\cdot$ &Punto medio (producto)\\ \hline
    \verb|\oplus| & $\oplus$ &Disyunción exclusiva\\ \hline
    \verb|\pm| & $\pm$ &Más menos\\ \hline
    \verb|\mp| & $\mp$ &Menos más\\ \hline
    \verb|\forall| & $\forall$ &Para todo\\ \hline
    \verb|\exists| & $\exists$ &Existe\\ \hline
    
    \end{tabular}
    \label{tab:binOperators}
    \caption{Símbolos matemáticos de relaciones binarias}
\end{table}
Todos los símbolos pueden ser negados (tachados con una línea ligeramente 
oblicua) si se preceden del operador \verb|\not|.
Por ejemplo \verb|\not\in| producirá $\not\in{}$.
El comando \texttt{not}, sin embargo, no vale para tachar partes de fórmulas,
para eso usaremos el paquete cancel, que permite tanto tachar como indicar
que una fórmula se cancela a un valor:
$$
\frac{225}{150} = 
\frac{\cancel{15}\cdot 15}{\cancel{15}\cdot 10}=
\frac{15}{10} = 
\frac{3\cdot \cancel{5}}{2\cdot \cancel{5}} = \frac{3}{2}
$$
$$
\frac{1}{\sqrt{2}} =
\frac{\sqrt{2}}{\cancelto{2}{\sqrt{2}\sqrt{2}}}\:\:\:\:= 
\frac{\sqrt{2}}{2}
$$

Dicho paquete -cancel- ofrece 4 órdenes:
\begin{itemize}
\item \verb|\cancel{<fórmula a tachar>}|: tacha la fórmula con una barra 
inclinada hacia delante: $\cancel{y}$
\item \verb|\bcancel{<fórmula a tachar>}|: tacha con una barra hacia atrás:
$\bcancel{y}$
\item \verb|\xcancel{<fórmula a tachar>}|: tacha con los dos comandos
anteriores. $\xcancel{y}$
\item \verb|\cancelto{<valor al que cancela>}{<fórmula a tachar>}|:
$3+3=\cancelto{6}{5}$
\end{itemize}

También están disponibles las letras griegas, tanto mayúsculas como minúsculas:

\begin{table}[H]
    \centering
    \begin{tabular}{|c|c|c|c|}
    \hline
    \textbf{Letra} & \textbf{Mayúscula} &
        \textbf{Minúscula} & \textbf{Presentación} \\ \hline

    alfa & \verb|\alpha| &
        \texttt{A} & $\alpha\, A$ \\ \hline

    beta & \verb|\beta| &
        \texttt{B} & $\beta\, B$ \\ \hline

    gamma & \verb|\gamma| &
        \verb|\Gamma| & $\gamma\,\Gamma$ \\ \hline

    delta & \verb|\delta| &
        \verb|\Delta| & $\delta\,\Delta$ \\ \hline

    épsilon & \verb|\epsilon| &
        \texttt{E} & $\epsilon\,E$ \\ \hline

    varépsilon & \verb|\varepsilon| &
        --& $\varepsilon$ \\ \hline

    zeta & \verb|\zeta| &
        \verb|\Zeta| & $\zeta\,Z$ \\ \hline

    eta & \verb|\eta| &
        \verb|\Eta| & $\eta\,H$ \\ \hline

    teta & \verb|\tetha| &
        \verb|\Theta| & $\theta\,\Theta$ \\ \hline

    varteta & \verb|\vartetha| &
        --& $\vartheta$ \\ \hline

    iota & \verb|\iota| &
        \texttt{I} & $\iota\,I$ \\ \hline

    kappa & \verb|\kappa| &
        \verb|\Kappa| & $\kappa \, K$\\ \hline

    lambda & \verb|\lambda| &
        \verb|\Lambda| & $\lambda\,\Lambda$ \\ \hline

    mu & \verb|\mu| & \texttt{M} & $\mu\,M$ \\ \hline

    nu & \verb|\nu| & \texttt{N} & $\nu\,N$ \\ \hline

    xi & \verb|\xi| &
        \verb|\Xi| & $\xi\,\Xi$ \\ \hline

    o & \texttt{o} & \texttt{O} &
        $o\,O$ \\ \hline

    pi & \verb|\pi| &
        \verb|\Pi| & $\pi\,\Pi$ \\ \hline

    varpi & \verb|\varpi| &
        --& $\varpi$ \\ \hline

    rho & \verb|\rho| &
        \texttt{P} & $\rho\,P$ \\ \hline

    varrho & \verb|\varrho| &
        --& $\varrho$ \\ \hline

    sigma & \verb|\sigma| &
        \verb|\Sigma| & $\sigma\,\Sigma$ \\ \hline

    varsigma & \verb|\varsigma| &--& $\varsigma$ \\ \hline

    tau & \verb|\tau| & \texttt{T} & $\tau\,T$ \\ \hline

    upsilon & \verb|\upsilon| &
        \verb|\Upsilon| & $\upsilon\,\Upsilon$ \\ \hline

    fi & \verb|\phi| &
        \verb|\Phi| & $\phi\,\Phi$ \\ \hline

    varfi & \verb|\varphi| &
        --& $\varphi$ \\ \hline

    chi & \verb|\chi| &
        \texttt{X} & $\chi\,X$ \\ \hline

    psi & \verb|\psi| &
        \verb|\Psi| & $\psi\,\Psi$ \\ \hline

    omega & \verb|\omega| &
        \verb|\Omega| & $\omega\,\Omega$ \\ \hline

    \end{tabular}
    \label{tab:greekLetters}
    \caption{Letras griegas}
\end{table}
Hay, además, un surtido de símbolos tipo flecha que se basan en la siquiente
regla: se escribe, seguida, la dirección de origen de la flecha,
después la de llegada, y si se pone la primera letra en mayúsculas, será una
flecha doble:
\begin{itemize}
\item \verb|\rightarrow| genera $\rightarrow$ (también disponible como
\verb|\to|).
\item \verb|\Rightarrow| genera $\Rightarrow$.
\item \verb|\leftarrow| genera $\leftarrow$.
\item \verb|\lefttarrow| genera $\Leftarrow$.
\item \verb|\leftrightarrow| genera $\leftrightarrow$.
\item \verb|\Leftrightarrow| genera $\Leftrightarrow$.

...
\end{itemize}
Y, finalmente, existen símbolos de varios tipos útiles como: el infinito,
verdadero y falso lógico, existe, para todo...
\begin{table}[H]
   \centering
\begin{tabular}{|c|c|c|}
    \hline
    \textbf{Comando}&\textbf{Presentación}&\textbf{Descripción} \\ \hline
    \verb|\infty| & $\infty$ & Símbolo del infinito\\ \hline
    \verb|\forall| & $\forall$ &Para todo \\ \hline
    \verb|\exists| & $\exists$ & Existe\\ \hline
    \verb|\top| & $\top$ &Verdadero lógico \\ \hline
    \verb|\bot| & $\bot$ &Falso lógico \\ \hline
    \verb|\lnot| & $\lnot$ &Negación lógica\\ \hline
    \verb|\partial| & $\partial$ &Derivada parcial \\ \hline
    \verb|\mathcal{L}| & $\mathcal{L}$ &Tranformada de Laplace \\ \hline
    \verb|\mathcal{F}| & $\mathcal{F}$ &Tranformada de Fourier \\ \hline
    \verb|\propto| & $\propto$ &Es proporcional a \\ \hline
    \verb|\dots| & $\dots$ &Puntos suspensivos \\ \hline
    \verb|\ddots| & $\ddots$ &Puntos diagonal\\ \hline
    \verb|\vdots| & $\vdots$ &Puntos verticales\\ \hline
    \verb|\cdots| & $\cdots$ &Puntos centrales\\ \hline
\end{tabular}
\caption{Símbolos matemáticos varios}
\label{tab:miscSymbols}
\end{table}

Además, es común que en matemáticas se usen acentos especiales para letras u
objetos, por ejemplo una flecha que simboliza que lo que hay debajo es un 
vector, éstos se ponen utilizando la orden correspondiente a los mismos y 
pasando como argumento el texto que ha de llevar el acento, por ejemplo:
\begin{verbatim}
$$
\vec{v}\times\vec{w} = \left| \matrix{i & j & k \cr
                                      v_x & v_y & v_z \cr
                                      w_x & w_y & w_z } \right|
$$
\end{verbatim}
generaría:
$$
\vec{v}\times\vec{w} = \left| \matrix{i & j & k \cr
                                      v_x & v_y & v_z \cr
                                      w_x & w_y & w_z } \right|
$$
Los acentos más comunes son:
\begin{table}[H]
\centering
\begin{tabular}{|c|c|}
    \hline
    \textbf{Comando}&\textbf{Ejemplo}\\\hline
    \texttt{vec} & $\vec{v}$ \\\hline
    \texttt{bar} & $\bar{v}$ \\\hline
    \texttt{hat} & $\hat{v}$ \\\hline
    \texttt{widehat} & $\widehat{v,\,w}
                                 \vphantom{\matrix{ \cr \cr}}$ \\\hline
    \texttt{acute} & $\acute{v}$ \\\hline
    \texttt{grave} & $\grave{v}$ \\\hline
    \texttt{dot} & $\dot{v}$ \\\hline
    \texttt{tilde} & $\tilde{v}$ \\\hline
\end{tabular}
\caption{Acentos posibles en ecuaciones}
\label{fig:mathAccents}
\end{table}

Nótese que el acento de barra no sirve para poner una barra encima de textos
largos, esto debe hacerse con el comando \verb|\overline|.
Por ejemplo:
\begin{verbatim}
$$
p\land q \equiv \overline{\left(\bar{p}\lor\bar{q}\right)}
$$
\end{verbatim}
genera (nótese el uso conjunto de \texttt{bar} y \texttt{overline}):
$$
p\land q \equiv \overline{\left(\bar{p}\lor\bar{q}\right)}
$$
Existe, además, el comando \texttt{underline} para subrayar en modo matemático.

Ya para ir terminando los comandos básicos de matemáticas, existen maneras de
añadir espacios a las ecuaciones, éstos serían:
\begin{table}[H]
    \centering
    \begin{tabular}{|c|c|c|}
        \hline
        \textbf{Comando}&\textbf{Ejemplo}&\textbf{Comentario}\\\hline
        \verb|<nada>|&$||$&Espacio normal\\\hline
        \verb|\,|&$|\,|$&Pequeño\\\hline
        \verb|\:|&$|\:|$&Mediano\\\hline
        \verb|\;|&$|\;|$&Grande\\\hline
        \verb|\!|&$|\!|$&Negativo\\\hline
    \end{tabular}
    \caption{Comandos de espaciado en ecuaciones}
    \label{tab:spaceEq}
\end{table}

Es remarcable el uso del espacio negativo presentado en la tabla
\ref{tab:spaceEq}. Uno de los usos más comunes es para la presentación de
números combinatorios. Cuando los mismos se escriben en forma matricial
(con el comando \verb!matrix!), se crean paréntesis muy lejos de los números:
\verb!\left(\matrix{7\cr3}\right)!
$$
\left(\matrix{7\cr3}\right)
$$
sin embargo, si lo hacemos
con el espacio negativo:\verb|\left(\!\matrix{7\cr3}\!\right)|
$$
\left(\!\matrix{7\cr3}\!\right)
$$

\subsubsection{Matrices como herramienta}\label{sec:matrixTools}
Es evidente que el uso principal del comando \texttt{matrix} es crear matrices
que contengan elementos algebraicos, pero en general nos permite crear rejillas
o tablas en nuestras ecuaciones que pueden resultar útiles para muchas cosas, 
por ejemplo, el siguiente código:
\begin{verbatim}
$$
|x|=\left\{ \matrix{    -x & \textrm{para} & x<0 \cr
                         x & \textrm{para} & x \geq 0}\right.
$$
\end{verbatim}
nos permitiría definir el valor absoluto como una función por partes, es decir:
$$
|x|=\left\{ \matrix{
-x&\textrm{para}&x<0\cr
x&\textrm{para}&x\geq0}\right.
$$
Hay varias cosas que comentar sobre este código:
\begin{enumerate}
\item El comanto \texttt{textrm} permite introducir texto (sin cursiva) en
ecuaciones.
\item Como se puede ver, hemos usado una matriz de dos filas y tres columnas 
para alinear los elementos, de este modo, en todas las filas la palabra <<para>>
está alineada con las demás.
\end{enumerate}

Aparte de este uso de las matrices, hay que tener en cuentra situaciones en las
que una matriz debe tener símbolos dentro, como por ejemplo la expresión
abreviada de un sistema de ecuaciones lineales.

$$
\left(
    \left.
        \matrix{
            3   &   2        &   1   \cr
            2   &   2        &   4   \cr
           -1   &\frac{1}{2} &  -1
        }
    \right|
    \matrix{
            1\cr
           -2\cr
            0
    }
\right)
$$
El código que se ha utilizado es este:
\begin{lstlisting}[language={[LaTeX]Tex}, caption={Ejemplo de matriz con barra},
label={lst:barMatrix}]
\left(
    \left.
        \matrix{
            3   &   2        &   1   \cr
            2   &   2        &   4   \cr
           -1   &\frac{1}{2} &  -1
        }
    \right|
    \matrix{
            1\cr
           -2\cr
            0
    }
\right)
\end{lstlisting}

Lo que se ha hecho aquí es crear una matrix de $3\times3$, sin delimitador
izquierdo y con una barra vertical como delimitador derecho (líneas 2-8),
después se ha creado una matrix de 3 filas con una columa justo al lado y
se ha rodeado todo con paréntesis con \verb!\left! y \verb!\right!.

Otro de los ejemplos que hay que utilizar es cuando se necesita una matriz 
rodeada de paréntesis con elementos externos, es decir:

$$
\matrix{
        a_{x,1} & a_{x,2} & a_{x,3}\cr 
        \left(
        \matrix{
            1  \cr
            4  \cr
            7
        }\right. &
        \matrix{
            1  \cr
            4  \cr
            7
        } &
        \left.
        \matrix{
            1  \cr
            4  \cr
            7
        }
        \right)
}
\!\!
\matrix{\cr a_{1,y}\cr a_{2,y}\cr a_{1,y}}
$$
\begin{minipage}[c]{\hsize}
\begin{lstlisting}[language={[LaTeX]Tex}, caption={Ejemplo de matriz con una 
matriz dentro},
label={lst:matrixOfMatrix}]
\matrix{
        a_{x,1} & a_{x,2} & a_{x,3}\cr 
        \left(
        \matrix{
            1  \cr
            4  \cr
            7
        }\right. &
        \matrix{
            1  \cr
            4  \cr
            7
        } &
        \left.
        \matrix{
            1  \cr
            4  \cr
            7
        }
        \right)
}
\matrix{\cr a_{1,y}\cr a_{2,y}\cr a_{1,y}}
\end{lstlisting}
\end{minipage}

El principal obstáculo que debemos superar para conseguir esto es que
no se pueden utilizar los operadores \verb!\left(! y \verb!\right)! entre
elementos de una fila de una matriz (es decir, que estén separados por un
\textit{et} (\&). Por esto, lo que hemos hecho es definir la matriz como una
matriz de $2\times 3$ (línea 1), donde la primera fila (línea 2) van a ser
los elementos <<externos>> y la segunda todos los demás,
rodeados de paréntesis. En la segunda fila de la matriz pondremos tres matrices
que, a su vez, tendrán tres filas. Es decir: la primera columna
de la matriz entre paréntesis está definida entre las líneas 4 y 8, la segunda
entre las líneas 9 y 13 y la última entre las líneas 15 y 19
Finalmente, hay que poner los paréntesis, esto se hará rodeando el primer
elemento de la segunda fila con un paréntesis por la izquierda
(orden left de la línea 3)  y el último
con uno por la derecha (orden right de la línea 20). Finalmente, después de
todo esto se añade una matrix de $4\times 1$, donde la primera fila está vacía
(línea 22).

Además, en matemáticas se usan fuentes especiales para indicar cosas, por 
ejemplo, una L mayúscula manuscrita es el símbolo de la transformada de Laplace.
Hay dos comandos que crean estas fuentes:
\begin{itemize}
\item \verb|mathcal\{ABCDEFG}|: Genera letras escritas a mano
(sólo vale con mayúsculas)\\$\mathcal{ABCDEFG}$
\item \verb|mathbb\{PNZQRC}|: Genera letras de doble trazo,
requiere el paquete amsfonts, incluido en el preámbulo: $\mathbb{PNZQRC}$.
\end{itemize}
Como es lógico, si sólo se requiere una letra, sólo ha de escribirse esa letra
entre las llaves.

Podemos poner símbolos monetarios gracias a algunos paquetes especiales
(ver comentarios en donde se incluyen los paquetes). Esta camisa cuesta
\EUR{10,99}. Para poner un dólar se pone una barra inclinada inversa 
<<\textbackslash>> y el signo de dólar. El resultado es: \$. Por ejemplo:
Esta mañana me he ido de compras por Nueva York y he gastado 500 \$.
\subsubsection{Paquete \texttt{amsmath}} \label{sec:amsmath}
El paquete \verb!amsmath! es un paquete que añade soporte a múltiples artefactos 
matemáticos avanzados, como macros para crear matrices con los símbolos
incorporados y hacer más fácil la creación de matrices más complejas como 
las que se han visto en \ref{sec:matrixTools}. El problema de este paquete es
que es \textbf{incompatible} con todo lo realizado hasta ahora en ecuaciones, 
por dos motivos, el primero es la notación utilizada para los bloques
matemáticos hasta ahora.

La notación que se ha visto para los bloques matemáticos (con signos de dólar)
es un legado de TeX (que es el origen de LaTex, que es un conjunto de macros
sobre él). En el LaTeX moderno se pueden denotar las ecuaciones con \verb!\[ \]!
y \verb!\( \)!, siendo el primero equivalente a los dos signos de dólar y el
segundo a único simbo de dólar. Es decir
\begin{verbatim}
$$
1+2=3
$$
\end{verbatim}
es equivalente a 
\begin{verbatim}
\[
1+2=3
\]
\end{verbatim}
Y \verb!\(1+2=3\)! lo es a \verb!$1+2=3$!. Esta es la notación que
\texttt{amsmath} requiere para funcionar. Además, \verb!amsmath!
no funciona con el comando \verb!\matrix! sino que requiere el uso de un entorno
que él mismo
proporciona, llamado del mismo modo. En este entorno debe usarse \verb|\\|
en lugar de \verb!\cr! para indicar la siguiente fila de una matriz, lee el
código del documento \verb!amsmath.tex! para ver los ejemplos,
este documento
no se compila automáticamente con \texttt{make}, sino que se debe compilar
a mano con \verb!pdflatex! o con el comando \verb!make amsmath!. Que generará
un PDF distinto al de este tutorial.

Alguna de las funcionalidades que quiero destacar de las
que ofrece este paquete son las siguientes:
\begin{enumerate}
\item Permite crear matrices con delimitadores incorporados sin el uso de
\verb!\left! y \verb!\right!.
\item Incorporación de los comandos \verb!\underbrace! y \verb!\overbrace!
que permiten dibujar una llave debajo o encima de un bloque con un texto
indicativo.
\item Permite rodear partes de una ecuación con un cuadrado con el comando
\verb!\boxed!.
\item incorporación de los comandos \verb!\int!, \verb!\iint! e \verb!\iiint!
que permiten integrales anidadas más atractivas. Y junto con ellas, el comando
\verb!\limits!, que permite que se sitúen los límites de las integrales encima
y debajo del símbolo.
\end{enumerate}
El paquete tiene muchas más funciones que pueden ser estudiadas en su
\href{https://texdoc.org/serve/amsmath/0}{documentación}. Si se quieren
ver ejemplos simples de las funciones citadas enteriormente se puede ver el 
código del archivo \verb!amsmath.tex!

En resumen sobre este paquete: es muy útil para matemáticas muy avanzadas, pero
si no se va a usar intensamente ninguna de las funcionalidades del mismo y ya
se tienen muchas ecuaciones escritas, incluir este paquete es un cambio que 
nos obligará a reescribir nuestras ecuaciones y hay que tener esto en cuenta.

\newpage \section{Bloques especiales; figuras y tablas}
Hay varios tipos de bloques especiales que nos permiten insertar cosas
más complejas en el texto que párrafos y listas. Los que vamos a tratar aquí
son:
\begin{enumerate}
\item Creación de bloques de código.
\item Figuras (imágenes).
\item Tablas.
\end{enumerate}

Los bloques verbatim son bloques donde el texto se incluye sin tomar en 
cuenta comandos de latex, interpretándolo literalmente y en letra 
monoespaciada. Son los bloques que he utilizado para incluir los fragmentos
de código \LaTeX{} que se han ido viendo en este documento.
Se suelen usar para justo eso: introducir código, en \LaTeX{} o en otros
lenguajes.
Por ejemplo, este es un programa que dice hola al usuario por su nombre
en el lenguaje C++.

\begin{verbatim} 
#include<iostream>
int main(void){
    std::string nombre;
    std::cout << "Dime tu nombre." << std::endl;
    std::cin >> nombre;
    std::cout << "Hola, " << nombre << "!" << std::endl;
    return EXIT_SUCCESS;
}
\end{verbatim}

En caso de querer incluir texto de este tipo dentro de un párrafo, usaríamos
\verb!\verb|<comandos|!. Por ejemplo: <<puede insertar
una barra inclinada inversa con \verb|\textbackslash|>> Además, los caracteres
que se usan para delimitar a verb pueden ser una gran variedad, es decir:
\verb!\verb|<comandos>|! es equivalente a \verb|\verb!<comandos>!|. Esto último
es muy útil cuando tenemos que usar caracteres especiales dentro de un
gramento verb, por ejemplo, en los ejemplos que estado introduciendo yo en este
mismo párrafo.

Gracias a que hemos cargado el paquete babel español (ver preámbulo del código),
podemos utilizar \texttt{<}\texttt{<} y \texttt{>}\texttt{>} para poner
comillas latinas. Por ejemplo:
\begin{quote}
<<El día que la mierda tenga algún valor, los pobres nacerán sin culo>>

-- Gabriel García Mázquez
\end{quote}

% Dos guiones son un guion largo

\subsection{Inclusión de figuras y tablas}
Para poner imágenes hay que usar la orden 
\verb|includegraphics|. Esta orden tiene la siguiente sintaxis
% utilizo un bloque verbatim para que se vea en el PDF, pero sólo lo de dentro
% es la orden
\begin{verbatim}
    \includegraphics{<ruta a la imagen>}  
\end{verbatim}
La ruta a la imagen puede ser absoluta (desde el inicio del árbol de
directorios) o relativa (desde este directorio donde está el archivo). Además,
con el comando \texttt{graphicspath} se puede indicar una dirección
base para el comando anterior. Supongamos una estructura de directorios como
la que sigue:
\begin{verbatim}
.
|-- document.PDF
|-- document.tex
|-- img
    `-- gatito.jpg
\end{verbatim}

Podríamos ejecutar la orden \verb|\graphicspath{{img/}}|
y así sólo tendríamos que especificar el nombre de los archivos, no hace falta 
especificar extensiones. Nótese que hay una barra inclinada después del nombre
del directorio y que hay dos llaves, son necesarias ambas cosas. Esto es porque
podríamos añadir más directorios separados por comas:
\verb|\graphicspath{{img/},{images/}}| incluiría dos
directorios, uno llamado <<img>> y otro llamado <<images>>. Hay que ser
cauteloso, incluir un espacio después de la coma que separa los directorios
haría que no funcionase.
Además, esta
orden ha de ir en el \textbf{preámbulo} del texto.

Con la opción \texttt{scale} podemos hacerla más pequeña -o más grande-.
Pero las imágenes en \LaTeX{} no están hechas para insertarse así, sino en una
figura, que es lo que nos permite alterar su alienamiento respecto al texto
y demás propiedades. La siguiente figura está insertada con este código:
\begin{figure}[H] %esta figura es para impedir que se me divida el verbatim
\begin{verbatim}
1   \begin{figure}[H]
2       \centering 
3       \includegraphics[width=.333\hsize]{gatito}
4       \caption{Un gato blanco}
5       \label{fig:gatitoBlanco} 
6   \end{figure}
\end{verbatim}
\end{figure}
Vamos a analizar lo que hacen las líneas una a una:
\begin{enumerate}
   \item Crea una figura, que es un cuadro invisible para \LaTeX{}, permite
que lo que haya en ella no se divida y añadirle un título y un símbolo para
referenciarla internamente.
    \begin{itemize}
        \item la opción H (que va entre corchetes) indica a \LaTeX{} que la
        figura debe ir en el sitio en que la hemos escrito en el código, de no
        utilizar esta opción, \LaTeX{} podría decidir un lugar distinto.
    \end{itemize}
    \item Esta línea indica que la figura debe estar centrada.
    \item Esta línea es la que incluye la imagen propiamente dicha.
    \begin{itemize}
        \item La opción (entre corchetes)
            \verb|width=.333\hsize| hace que la imagen ocupe
            el 33,3 \% del ancho de la página. Si cambias el número ($.333$)
             por ejemplo a $0.5$ ocuparía el 50 \%.
        \item Como hemos usado la orden \texttt{graphicspath} en el preámbulo.
        Podemos escribir simplemente el nombre del archivo, sin indicar la ruta.
    \end{itemize}
    \item La orden \texttt{caption} pone el título visible que tendrá la figura.
    \item La orden label crea la etiqueta interna de esta figura, y permite 
    luego referenciarla. Esta orden debe ir \textbf{después} de
        \texttt{caption}. Cada tipo de elemento en LaTeX tiene su prefijo, por
        ejemplo, las figuras deben ser \texttt{fig:XXX} y las tablas 
        \texttt{tab:XXX} donde \texttt{XXX} es un nombre arbitrario.
    \item Finalmente, esto termina la figura.
\end{enumerate}

El resultado es el siguiente:
% [H] inserta aquí, t al inicio de la página, b al final... hay muchas opciones
    % Usa "H" si quieres que la figura esté en el mismo orden que como la has
    % escrito, es la opción más intuitiva.
    % (ver sección donde incluyo los paquetes)
% Gracias a que hemos puesto la orden \graphicspath{{./img/}} en el preámbulo
    % de nuestro documento, ahora sólo tenemos que poner el nombre de las
    % imágenes que estén ahí.
%inicio de una figura.
\begin{figure}[H]
    % Centra la imagen
    \centering 
    % Con width=1.0\hsize hacemos que la imagen sea tan ancha como el párrafo,
    % yo lo recomiendo para que las cosas queden alineadas, salvo que por x
    % motivo quieras una imagen más pequeña, entonces cambia 1.0 por otro nº.
    \includegraphics[width=.333\hsize]{gatito}
    % Título de la figura que se ve
    \caption{Un gato blanco}
    % Etiqueta interna, se usa para referencias la imagen. Esta orden debe ir
    % siempre después de la caption. Cada tipo de figura tiene su formato de 
    % label, por ejemplo las figuras son fig: y las tablas tab:, respétalo o 
    % LaTeX no podrá encontrarlas para los índices.
    \label{fig:gatitoBlanco} 
\end{figure}

Al insertar un párrafo aquí podemos ver cómo se organizan las imágenes, de tal
modo que se insertan en la página necesaria. Las figuras no se dividen a sí
mismas entre páginas.

\begin{figure}[H]
    %centra la imagen (por si no quedaba claro)
    \centering
    % El número antes de hsize es un multiplicador, por ejemplo, podemos hacer
    % que ocupe un tercio de ancho
    \includegraphics[width=0.3333\hsize]{gatitoNegro}
    \caption{Un gato negro} %caption que se ve
    \label{fig:gatitoNegro} %nombre interno para referenciar luego
\end{figure}

\subsubsection{Referencias cruzadas} \label{subsubsec:referencias cruzadas}
Una referencia cruzada es un punto en el texto en que hablamos de un elemento
del mismo, por ejemplo, si quisiéramos referenciar aquí la figura titulada
<<Un gato negro>>, podríamos ponerlo a mano en el texto: Como se puede ver
en la Figura 2: Un gato negro. Pero, ¿qué pasaría si cambiáramos ese título?
¿Y si incluyéramos una imagen antes de esta y su número cambiara? Tendríamos
que rastrear todas las veces que la hemos mencionado en el documento y
cambiarlo, lo cual sería tedioso y llamaría a errores frecuentes.

Para eso
usaremos el comando \texttt{ref}. Este comando nos permite referenciar una
figura (o cualquier objeto con una etiqueta). 

Si ahora quieres referenciar una figura en un párrafo en \LaTeX{} puedes poner
\verb|\ref{<nombre de la referencia>}|. Esto insertará el 
número de la figura. Si insertamos \verb|\nameref{<nombre de la referencia>}|
aparecerá el título de la misma (la \textit{caption}).

Por ejemplo: como podemos ver en la Figura \ref{fig:gatitoNegro}:
\nameref{fig:gatitoNegro}.
\subsubsection{Inclusión de gráficos vectoriales}
Aunque incluir fotografías es muy útil, incluir otro tipo de imágenes es
muy útil también. Uno de esos tipos de imagen más interesante son los gráficos
vectoriales. Estos gráficos son imágenes definidas en términos matemáticos,
esto permite que sean ampliados todo lo que se desee sin que aparezca el efecto
de <<borrosidad>> o pixelación que aparece en las imágenes normales.

\LaTeX{} no permite de manera fácil insertar gráficos vectoriales, que
generalmente se distribuyen en archivo SVG (del inglés: \textit{Scalable
Vector Graphics}), pero podemos trabajar para sortear esto. Podemos incrustar
un PDF. Si disponemos de un gráfico vectorial, debemos usar un programa de
nuestra elección para convertirlo a PDF, por ejemplo: InkScape. Una manera
fácil es abrir la imagen svg en el navegador web e imprimirla a una impresora
en PDF. Hay que tener cuidado con los márgenes que nos deje alrededor de la
imagen, porque existirán cuando la incrustemos en nuestro documento. De todos
modos, hay convertidores en internet. La cuestión es que debes convertir el SVG
a PDF. Si tienes necesidad de incluir este tipo de gráficos, es probable que 
seas un usuario avanzado.

Una vez tenemos nuestro archivo PDF, debemos incluirlo, y se hace del mismo
modo que cualquier otra imagen, con un entorno figure y la orden
\verb!includegraphics!. Por ejemplo, el dibujo de una casa que se ve en
la figura \ref{fig:house_svg} se ha creado convirtiendo el archivo SVG a PDF
con InkScape. Se puede descargar el programa tanto para Windows como para Linux
desde \href{https://inkscape.org}{aquí}.
\begin{figure}[H]
\centering
\includegraphics[width=.5\hsize]{house}
\caption{El dibujo vectorial de una casa}
\label{fig:house_svg}
\end{figure}

Para incrustarla se han usado, como se ha dicho, los mismos comandos que si el
PDF fuera una imagen, pero se incluye el código en el programa
\ref{lst:pdf_include} para explicitar esto.

\begin{lstlisting}[language={[LaTeX]TeX},
                    caption={Incrustación de PDF},
                    label={lst:pdf_include}]
\begin{figure}[H]
\centering
\includegraphics[width=.5\hsize]{house}
\caption{El dibujo vectorial de una casa}
\label{fig:house_svg}
\end{figure}
\end{lstlisting}
\subsection{Tablas}
Las tablas son una de las herramientas más complejas de \LaTeX{}, empezaremos
insertando tablas básicas (donde todas las celdas estén ocupadas) y después
veremos cómo combinarlas y añadir formato más elaborado.
A continuación, vamos a insertar la primera tabla:

% Esto no sirve para hacer la tabla en sí misma, sino para ponerle captions y
% esas cosas. Es equivalente a \begin{figure} que no inserta una imagen, pero
% nos prepara las cosas para que se comporte como una figura, así podemos
% ponerle título y referenciarla.
\begin{table}[H]
    \centering
    % Aquí defines la tabla en sí misma, tienes que poner tantas letras como
    % columnas vaya a tener la tabla, y puedes poner líneas verticales entre
    % ellas escribiendo este símbolo entre las letras. Estas letras son los
    % especificadores de alineación. Si pones l, la columna se alinea a la
    % izquierda, si pones c, al centro, y si pones r, a la derecha.
    \begin{tabular}{|c|r|c|r|}
    % Esta orden genera una línea entre filas de la tabla.
    \hline
    % En las tablas las celdas se separan con & y las filas con \\
    % Puede haber celdas vacías, pero todas las filas deben tener las mismas
    % celdas.
    Concepto & Precio Unitario & Cantidad & Subtotal\\ \hline
    Placa base & \EUR{89,99} & 1 & \EUR{89,99}\\ \hline
    RAM 8GB DDR4 3200 MHz & \EUR{40,44} & 4 & \EUR{161,76} \\ \hline
    % Aquí podemos ver que hemos puedo celdas vacías. (dos & seguidos)
    \textbf{Total}&&&\textbf{\EUR{251,75}} \\ \hline
\end{tabular}
    % De nuevo, el título que se ve
    \caption{Gastos de la reparación}
    % La referencia interna. 
    \label{tab:gastos}
\end{table}

La tabla ha sido creada de este modo:

\begin{verbatim}
1   \begin{table}[H]
2       \centering
3       \begin{tabular}{|c|r|c|r|}
4           \hline
5           Concepto & Precio Unitario & Cantidad & Subtotal\\ \hline
6           Placa base & \EUR{89,99} & 1 & \EUR{89,99}\\ \hline
7           RAM 8GB DDR4 3200 MHz & \EUR{40,44} & 4 & \EUR{161,76} \\ \hline
8           \textbf{Total}&&&\textbf{\EUR{251,75}} \\ \hline
9       \end{tabular}
10  \caption{Gastos de la reparación}
11  \label{tab:gastos}
12  \end{table}
\end{verbatim}

\begin{enumerate}
\item Inicia la <<figura tabla>>, esto es equivalente a cuando incrustamos
una figura que va a contener una imagen.
\item Hace que la tabla esté centrada, aclaración: cuando la tabla es más 
grande que el tamaño del texto se alinea a la izquierda y se desborda por la
derecha.
\item Empieza la tabla en sí misma, la segunda parte \texttt{\{|c|r|c|r|\}}
indica el número de celdas (que se corresponde con el número de letras) y las 
líneas verticales (|) indican que queremos una línea entre esas dos columnas.
Las líneas verticales son opcionales, pero los especificadores de alineación
deben aparecer en el mismo número que columnas queramos que tenga la tabla.
\item Dibuja una línea horizontal en toda la tabla
\item Es una fila de la tabla, el texto irá en la misma celda hasta que haya
un \textit{et} (\&), así se pasa de columna. Cuando se haya terminado la fila,
se deben poner dos líneas inclinadas invertidas 
(\verb|\\|). 
\item Siguiente fila, apréciese el uso de la orden 
\verb|\EUR{}|. (ver: \textbf{\ref{ecuaciones}})
\item La siguiente fila.
\item La última fila, nótese que se pueden poner celdas vacías poniendo
dos \emph{et} (\&\&) Y que debe terminarse con \verb|\\|.
\item Termina la tabla
\item Indica el título visible de la tabla.
\item Etiqueta para referencias.
\end{enumerate}

Además de tablas básicas, se pueden crear tablas más complejas, con celdas
combinadas, por ejemplo: a veces es deseable crear celdas que ocupen es
espacio de otras, en la tabla \ref{tab:animals} se ven celdas que ocupan
varias columnas.

\begin{table}[H]
    \centering
    \begin{tabular}{|c|c|c|c|}
        \hline
        \multicolumn{4}{|c|}{\textbf{Animales}} \\\hline
        \multicolumn{2}{|c|}{\textbf{Mamíferos}} & \multicolumn{2}{c|}{\textbf{Aves}}\\\hline
        Oso&Perro&Cigüeña & Gorrión \\\hline
        \multicolumn{1}{|r|}{500 Kg}&
            \multicolumn{1}{r|}{10 Kg}&
            \multicolumn{1}{r|}{6 Kg} &
            \multicolumn{1}{r|}{0,02 Kg} \\\hline 
    \end{tabular}
    \caption{Peso medio de animales}
    \label{tab:animals}
\end{table}

Para hacer esto se usa la orden \verb|\multicolumn|, que tiene esta sintaxis:
\begin{verbatim}
\multicolumn{<número de columnas>}{<alineación>}{<contenido>}
\end{verbatim}
El número de columnas no necesita explicación. La alineación indicará cómo
se alineará el texto de esta celda grande, funciona como las de la propia tabla,
es decir, se puede poner l, r o c y con barras a los lados (o sólo a uno), para
dividir las columnas. El contenido es el texto de la celda combinada. Por 
ejemplo, la celda donde pone <<Animales>> se ha creado con esta orden:
\verb!\multicolumn{4}{|c|}{\textbf{Animales}}!. Ocupa cuatro columnas, su
contenido está centrado, y tiene barras a ambos lados de la celda. Es
importante recordar que hay que omitir los et (\&) de las columnas
que se hayan omitido, es decir, al haber hecho una celda de 4 columnas con
la orden anterior, no se ha de poner ningún et.


Aparte de esto, quizá sea necesario que una celda ocupe varias filas, por
ejemplo en la tabla \ref{tab:multirows}.

\begin{table}[H]
    \centering
    \begin{tabular}{|c|c|c|c|c|c|}
        \cline{3-6}
        \multicolumn{2}{c|}{}&\multicolumn{4}{c|}{Pruebas} \\\cline{3-6}
        \multicolumn{2}{c|}{}&T-01&T-02&T-03&T-04\\\hline
        \multirow{4}{*}[-.2em]{\begin{sideways}Requisitos\end{sideways}}&REQ-01&OK&ERROR&OK&ERROR\\\cline{2-6} 
        &REQ-02&OK&OK&OK&OK\\\cline{2-6}
        &REQ-03&OK&OK&OK&OK\\\cline{2-6}
        &REQ-04&OK&ERROR&OK&ERROR\\\hline 
    \end{tabular}
    \caption{Trazabilidad -- Ejemplo de tabla con \emph{multirows}}
    \label{tab:multirows}
\end{table}

Esta tabla se ha creado con este código:

\begin{minipage}[c]{\hsize}
\begin{lstlisting}[language={[LaTeX]TeX},
                    caption={Código de celda multifila},
                    label={lst:multirowTable}]
\begin{table}[H]
    \centering
    \begin{tabular}{|c|c|c|c|c|c|}
        \cline{3-6}
        \multicolumn{2}{c|}{}&\multicolumn{4}{c|}{Pruebas} \\\cline{3-6}
        \multicolumn{2}{c|}{}&T-01&T-02&T-03&T-04\\\hline
        \multirow{4}{*}[-.2em]{\begin{sideways}Requisitos\end{sideways}}&REQ-01&OK&ERROR&OK&ERROR\\\cline{2-6} 
        &REQ-02&OK&OK&OK&OK\\\cline{2-6}
        &REQ-03&OK&OK&OK&OK\\\cline{2-6}
        &REQ-04&OK&ERROR&OK&ERROR\\\hline 
    \end{tabular}
    \caption{Trazabilidad -- Ejemplo de tabla con \emph{multirows}}
    \label{tab:multirows}
\end{table}
\end{lstlisting}
\end{minipage}

\begin{enumerate}
\item Inicio de tabla.
\item Centrar la tabla.
\item Empezar la definición de las celdas.
\item Primer comando nuevo: cline es equivalente a hline, pero permite
especificar un intervalo de celdas que tendrán la línea, en este caso,
queremos que sean las 3 últimas de la tabla, porque las dos primeras no las
queremos dibujar.
\item Comando multicolumn, se puede ver que se define una celda de dos columnas
con alineación central y sin barra dibujada a la izquierda (\verb!c|!), se
crea vacía, es la esquina que <<falta>> de la tabla.
\item Mismo procedimiento respecto al multicolumn para dejar esa parte de la
tabla en blanco.
\item Empieza la multifila:
    \begin{itemize}
        \item Definimos que va a ocupar cuatro filas.
        \item El siguiente argumento es el ancho de la columna, con un asterisco
indicamos que va a ser ajuste automático.
        \item El siguiente argumento (que es opcional) indica que el texto
ha de salir un poco más abajo que al inicio de la celda. Finalmente, indicamos
el contenido de la celda. Más adelante desarrollaré esto, de momento simplemente
pensar que el contenido de la celda se escribe ahí.
    \end{itemize}
\item Siguiente fila, nótese que las celdas <<ocupadas>> por el multifila deben
estar en blanco, pero sus símbolos \textit{et} deben aparecer.
\item Nótese el uso de cline en estas celdas.
\end{enumerate}

Para poner el texto girado 90 º a la izquierda se ha usado el paquete
\texttt{rotating}, que introduce el entorno \verb|sideways|, que escribe
el texto de este modo. Por ejemplo:
\begin{verbatim}
\begin{sideways}
Estoy girado.
\end{sideways}
\end{verbatim}

generaría el siguiente párrafo:

\begin{sideways}
Estoy girado.
\end{sideways}







De nuevo, podemos referenciar el número de la tabla con 
\verb|\ref{tab:gastos}| y su
títulos con \verb|\nameref{tab:gastos}|. \textbf{Nota sobre compilar 
referencias cruzadas:}
Cuando no se encuentra una referencia (por ejemplo porque se ha escrito
una etiqueta que no existe) salen signos de
interrogación en lugar de las referencias,
si te pasa y has comprobado que la referencia está bien escrita
 no te preocupes, \LaTeX{} a veces necesita dos compilaciones para crear
la base de datos de referencias. Compila otra vez para asegurarte que no es 
ése el problema.
En el makefile proporcionado ya se compila dos veces, pero si no lo utilizas,
simplemente llama al comando dos veces seguidas.

\subsection{Utilización alternativa de las figuras}
Una de las ventajas de utilizar figuras es que el contenido de las mismas no se
divide entre páginas, así que si quisiéramos que una lista en concreto no se
dividiera, podríamos incluirla en un bloque de figura, esta lista, por ejemplo:
\begin{figure}[H]
\begin{itemize}
   \item a
   \item b
   \item c
   \item d
   \item e
   \item f
   \item g
   \item h
   \item i
   \item j
   \item l
   \item m
   \item n
   \item ñ
   \item o
   \item p
   \item q
   \item r
   \item s
   \item t
   \item u
   \item v
   \item w
   \item x
   \item y
   \item z
\end{itemize}
\end{figure}
Esto debe usarse con cuidado, empero, pues generaría muchos espacios en blanco
en nuestro documento. Y, además, se causaría un espacio en blanco entre los
párrafos que estuvieran antes y después que la figura, si se desea evitar esto
se debe usar el entorno \emph{minipage}. Por ejemplo, para este caso de uso
se haría de este modo:

\begin{minipage}[c]{\hsize}
    \begin{lstlisting}[ language={[LaTeX]TeX},
                        caption={Uso de minipage para impedir división entre
                         páginas},
                        label=lst:styleCode]
\begin{minipage}[c]{\hsize}
\begin{itemize}
% items de la lista...
\end{itemize}
\end{minipage}
\end{lstlisting}
\end{minipage}

Si se usa la solución de \emph{minipage} debe tenerse cuidado porque el entorno
debe estar separado por una línea en blanco del párrafo anterior, o producirá
que la última línea del párrafo salga a la izquierda de la minipágina. Además, 
a veces ese espacio posterior que añade el bloque figure es deseable. Por todo
esto, el uso de una figura o de \emph{minipage} queda a discreción del
escritor.

\subsection{Bibliografía; referencias bibliográficas}
Además de referenciar otras partes de nuestro propio documento (ver
\textbf{\ref{subsubsec:referencias cruzadas}}) es común necesitar una lista
de referencias bibliográficas externas, para esto utilizaremos el paquete
BibTex, que nos permite añadir archivos de bibliografía donde incluiremos
todas las fuentes que necesitemos (podemos incluirlas en distintos archivos
si queremos, e incluir varios), después, nos permitirá citar en el texto esas
fuentes e insertar la sección de bibliografía.

Para crear bibliografía, hay que crear un fichero de extensión .bib, que
contendrá las citas bibliográficas que sean necesarias en un formato especial.
Por ejemplo, en este repositorio hay un directorio llamado bibliography, dentro
del cual hay un archivo llamado sources.bib. Este archivo contiene una fuente
bibliográfica para un libro titulado \emph{El Lenguaje de programación C++}
escrito por Bjarne Stroustrup. El contenido del archivo .bib es el siguiente:
\begin{verbatim}
@book{cpppl,
author = {Stroustrup, Bjarne},
title = {The C++ Programming Language},
year = {2013},
isbn = {0321563840},
publisher = {Addison-Wesley Professional},
edition = {4th},
abstract = {<Resumen del libro, omitido porque es muy largo>}
}
\end{verbatim}

Es decir, para crear un recurso bibliográfico se debe poner una arroba <<@>>,
el tipo de recurso (libro, artículo...), se abre una llave, se pone una 
\textbf{etiqueta}, que es el primer texto: \texttt{cpppl}. Esta es la etiqueta
que usaremos para citar el recurso en el texto.

Hay varios tipos predeterminados de documentos en el paquete biblatex que nos 
permiten citar automáticamente, algunos de ellos son: article, book, thesis,
masterthesis... Pero la principal ventaja del formato bibtex es que muchísimas
plataformas de acceso online permiten copiar directamente el texto que debes
incluir en tu archivo .bib. Más información sobre esto más adelante.

Para citar el libro, utilizaremos el comando
\verb|\cite{<etiqueta de la fuente>}|. Veamos un ejemplo, el
código siguiente:
\begin{verbatim}
<<El lenguaje C++ es un superconjunto del lenguaje C>> (\cite[p.~101]{cpppl})
\end{verbatim}
genera este párrafo:

<<El lenguaje C++ es un superconjunto del lenguaje C>> (\cite[p.~101]{cpppl})
Nótese que los paréntesis se han de añadir aparte, y que entre el p. y el
número de página ha de haber una tilde (\~{}), para añadir un espacio. El estilo
de la cita viene dado por las opciones del paquete bibtex incluido en el 
preámbulo, en este caso se ha elegido el estilo de citación APA. Algunos
de los estilos permitidos son:
\begin{table}[H]
\centering
\begin{tabular}{|c|c|}
\hline
    \textbf{Estilo} & \textbf{Argumento} \\ \hline 
    ACS & chem-acs\\\hline
    AIP & phys\\\hline
    Nature& nature\\\hline
    Science & science\\\hline
    IEEE & iee\\\hline
    Chicago& chicago-authordate\\\hline
    MLA & mla\\\hline
    APA & apa\\\hline
\end{tabular}
\caption{Estilos de citación permitidos por el paquete biblatex}
\label{tab:biblatexStyles}
\end{table}
Para elegir estos estilos, debe irse a la línea del preámbulo donde hayamos
incluido el paquete biblatex (recordemos, con la orden usepackage) y modificar
los argumentos de las opciones \texttt{style} y \texttt{citestyle}. Por 
ejemplo, la línea actual es la siguiente:

\begin{verbatim}
\usepackage[backend=biber, style=apa, citestyle=apa]{biblatex}
\end{verbatim}
Si quisiéramos cambiar el estilo para ajustarse al del IEEE, se cambiaría a:
\begin{verbatim}
\usepackage[backend=biber, style=ieee, citestyle=ieee]{biblatex}
\end{verbatim}

Además de citar, necesitaremos imprimir nuestra bibliografía completa en
nuestro documento, para ello, usaremos el comando \verb|\printbibliography|.
Este comando inserta un título predeterminado que, en 
español (recordemos, paquete babel spanish) es <<Referencias>>, que, además,
no sale en la tabla de contenido. Si queremos omitirlo, podemos añadir la opción
\texttt{heading=none} en la orden, es decir:

\begin{verbatim}
\printbibliography[heading=none]
\end{verbatim}
e incluir nosotros un título de la jerarquía que deseemos justo antes del 
comando. Esto es lo que se ha hecho en este documento. 

Por otro lado, algunas organizaciones exigen que las citas bibliográficas se 
incluyan en notas al pie allí donde se citen. Por suerte \LaTeX{} permite 
crear notas al pie de manera sencilla, simplemente se utiliza el comando
\verb|\footnote{<nota al pie>}|. Veamos un ejemplo.

\begin{verbatim}
Al final de esta frase va a haber una nota al pie \footnote{Esta es la nota
al pie.}.
\end{verbatim}
genera el siguiente párrafo y su nota el pie correspondiente:

Al final de esta frase va a haber una nota al pie \footnote{Esta es la nota
al pie.}.

Si queremos que en la nota el pie aparezca la cita, simplemente incluimos dentro
del comando \texttt{footnote} el comando \texttt{cite}. Además, tenemos la
opción de incluir en el texto la cita completa (tal y como aparecería en la
bibliografía, con el comando \verb|\fullcite{<etiqueta>}|. Veamos un ejemplo
de una nota el pie con el texto completo de la cita; este código:
\begin{verbatim}
La metaprogramación es una de las características más importantes del lenguaje
C++ \footnote{\fullcite{cpppl}}.
\end{verbatim}
generaría:

La metaprogramación es una de las características más importantes del lenguaje
C++ \footnote{\fullcite{cpppl}}.

\subsubsection{Importar referencias en formato Bibtex}
Como se ha dicho en la sección anterior, muchos servicios permiten importar las
citas bibliográficas directamente en el formato bibtex.
Por ejemplo, busquemos el libro \emph{Historia de la
decadencia y caída del Imperio Romano} en Google Scholar, lo encontramos en
este \href{https://books.google.es/books?id=uydf0KUHpeoC}{enlace}.

Justo al final de esa página hay un botón que nos permite exportar la cita
bibliográfica a varios formatos:

\begin{figure}[H]
    \centering
    \includegraphics[width=0.75\hsize]{exportarBibtex}
    \caption{Ejemplo de exportar desde Google Books}
    \label{fig:exportGoogle}
\end{figure}

Si elejimos el formato bibtex, nos descargará un archivo con la referencia
que necesitamos, esto nos permitirá copiar su contenido a nuestro archivo
de bibliografía (en este caso, sources.bib). Aunque es recomendable guardar
el archivo descargado aparte por si queremos utilizarlo en otros documentos
a modo de biblioteca personal de citas, en algún lugar de nuestro sistema.

Ahora lo podemos citar:

<<El imperio romano cayó, en parte, por...>> (\cite{gibbon2016historia})

Y se añadirá a nuestra colección de citas en la sección de bibliografía. Bibtex
tiene la ventaja de que aunque en nuestras fuentes bibliográficas haya fuentes
que no hemos citado, no va a incluirlas hasta que las citemos.

\textbf{Nota sobre el formato}: Algunos servicios de documentación, como
el propio Google Books, eliminan los caracteres especiales de sus citas
exportadas (las tildes, la eñe...) y los ponen por ejemplo: 
\verb|{\`i}|
en vez de: í. Esto provoca errores, así que simplemente debe cambiarse por
el carácter correspondiente. Otras permiten elegir la codificación, tal
y como está configurado este documento, debe elegirse UTF8. Y debe revisarse
que las etiquetas no contengan estos caracteres especiales de ningún modo, lo
ideal sería sustituirlos por sus equivalentes sin acentos, u por ü, i por í,
etc.

\textbf{Nota sobre la compilación}: Debido a cómo funciona el motor bibtex, 
para que las referencias bibliográficas se creen de manera adecuada, primero
debe compilarse con \texttt{pdflatex}, después, llamar a comando \texttt{biber}
sobre el archivo y después compilar dos veces. En el makefile proporcionado
esto ya se ha hecho, además, utilizando opciones de estos comandos para
que los archivos auxiliares se creen en la carpeta build.

\textbf{Nota para los usuarios de TexStudio}: En este programa, para generar
las fuentes bibliográficas, se debe acudir al menú:
\texttt{Herramientas$\rightarrow$Órdenes$\rightarrow$Biber}
para que se genere la bibliografía, y se debe hacer \textbf{después} de 
compilar una primera vez.

\subsection{Entornos -\emph{enrironments}- especiales}
En esta sección enumeraré y explicaré bloques especiales que se pueden
utilizar para cosas más concretas. Requieren la inclusión de más paquetes y,
además, cumplen propósitos muy específicos.
\subsubsection{Código con coloreado de sintaxis (\textit{listings})}
El paquete \textit{listings} permite incluir un bloque parecido al bloque
\textit{verbatim}, pero con la capacidad de colorear de una manera comprensible
el código en diferentes lenguajes, añadir números de línea y definir estilos
con marcos, colores de fondo y estilos de letra. La configuración de este
paquete tiene tres pasos:
\begin{enumerate}
\item Incluir el paquete (como todos los demás).
\item Definir el estilo del código.
\item Definir el nombre de los fragmentos de código que incluyamos (esto es
porque queremos que salga en español, predeterminadamente es \textit{listings})
\item Definir el nombre de la lista de fragmentos de código, esto es por el
mismo motivo que el punto anterior.
\item Incluir algunos fragmentos.
\end{enumerate}

Para definir el estilo del código vamos a usar la orden \verb|\lstset{...}|,
por ejemplo, el estilo que se ha creado aquí se ha tipificado así:

\begin{minipage}[c]{\hsize}
\begin{lstlisting}[ language={[LaTeX]TeX},
                    caption={Especificación del estilo de \textit{listings}},
                    label=lst:styleCode]
\lstset{
    basicstyle=\ttfamily,
    numberstyle=\ttfamily,
    frame=rlbt, % draw a frame at the top and bottom of the code block
    tabsize=4, % tab space width
    showstringspaces=false, % don't mark spaces in strings
    numbers=left, % display line numbers on the left
    commentstyle=\color{gray}, % comment color
    keywordstyle=\color{blue}, % keyword color
    stringstyle=\color{red}, % string color
    backgroundcolor=\color{lightYellow},
    captionpos=b,
    breaklines=true,
    literate=
            {Á}{{\'{A}}}1
            {É}{{\'{E}}}1
            {Í}{{\'{I}}}1
            {Ó}{{\'{O}}}1
            {Ú}{{\'{U}}}1
            {á}{{\'{a}}}1
            {é}{{\'{e}}}1
            {í}{{\'{i}}}1
            {ó}{{\'{o}}}1
            {ú}{{\'{u}}}1
            {ñ}{{\~{n}}}1
            {Ñ}{{\~{N}}}1
            {ü}{{\"{u}}}1
            {Ü}{{\"{U}}}1
            {¡}{{!`}}1
            {¿}{{?`}}1
}
\end{lstlisting}
\end{minipage}
Las líneas hacen lo siguiente:
\begin{enumerate}
\item Inicia el comando para definir el estilo.
\item Indica que el texto de los \textit{listings} debe salir en fuente
monoespaciada (o de máquina de escribir).
\item Indica que los números de línea deben salir también en fuente
monoespaciada.
\item El argumento frame puede recibir un subconjunto de $\{t,r,b,l\}$,
que define qué bordes del cuadrado que lo engloba se dibujan. El orden en que
se ponen las letras es irrelevante.
\begin{enumerate}
\item t es por \textit{top}, arriba.
\item r es por \textit{right}, derecha.
\item b es por \textit{bottom}, abajo.
\item l es por \textit{left}, izquierda.
\end{enumerate}
\item Indica el tamaño del carácter tabulador, si tabulas con espacios no 
tiene efecto.
\item Indica si los espacios en las cadenas de caracteres debe mostrarse, es
decir, mostraría algo como <<''$\textrm{Hola,␣} \textrm{mundo!}$''>>
en vez de <<''Hola, mundo!''>>, puede ser \textit{true} o
\textit{false}.
\item La opción \textit{numbers} puede ser: \textit{left},
\textit{right}, \textit{none}. Para mostrar los números a izquierda o derecha
del código; o no mostrarlos.
\item Define el estilo de los comentarios del lenguaje, aquí se ha especificado
únicamente que se cambie su color a gris. Más información sobre colores en 
\textbf{\ref{coloreado}}.
\item Define el estilo de las palabras clave de los lenguajes, por ejemplo, en
C, serían palabras clave: int, double, void, char...
\item Define el estilo de las cadenas de caracteres en el lenguaje.
\item Define un color de fondo, en este caso un color personalizado amarillo
claro.
\item El argumento \texttt{captionpos} puede ser b (por \textit{bottom}, abajo)
o t (por \textit{top}, arriba), indica dónde ha de aparecer la \textit{caption}.
\item Indica si las líneas más largas deben romperse automáticamente o no,
si se deja en \textit{false}, se comportará como los bloques
\textit{verbatim} y los desbordará por la derecha si la línea es muy larga.
\item A partir de aquí estas líneas sirven para que cuando lstlisting se
encuentre un carácter del especial del español lo sepa procesar. Si necesitas
añadir más caracteres simplemente añade una línea exactamente igual pero cambia
la letras (por ejemplo Á) y el comando. 
\end{enumerate}

Para definir el nombre de los fragmentos de código que incluyamos debemos usar
la orden \verb|\renewcommand{\lstlistingname}{<nombre>}|, donde 
\texttt{<nombre>} es el nombre que queramos darle a los fragmentos de código.
Después, para definir el nombre de la lista de fragmentos, podemos utilizar
\verb|\renewcommand{\lstlistlistingname}{Lista de \\MakeLowercase{\lstlistingname s}}|.
De este modo, la lista quedará configurada con el nombre <<Lista de Programas>>.


Para incluir algunos fragmentos se hace, como con otros \textit{environments}, 
con las órdenes \verb|\begin| y \verb|\end|. Además, el paquete recibe algunas
opciones en la orden begin que nos permiten indicar el lenguaje (para que se
destaquen las palabras clave automáticamente), la \textit{caption} (el título
visible que saldrá en el documento) y la etiqueta (que será la que usemos para
referencias). Por ejemplo:
\begin{lstlisting}[language={[LaTeX]TeX},mathescape=true]
\begin{lstlisting}[ language={c++},
                    caption={Ejemplo de Hola Mundo en C++},
                    label=lst:helloWordCpp]
\end{$ $lstlisting}
\end{lstlisting}


Definirá el lenguaje, el título y la etiqueta del siguiente fragmento de código,
nótese que no se incluye ni el cierre del entorno \verb|\lstlisting| ni el
código C++ del ejemplo.
\begin{lstlisting}[language=c++, caption={Ejemplo de Hola Mundo en C++},
label=lst:helloWordCpp]
#include<iostream>
int main(int argc, const char* argv[]){
    std::string nombre;
    std::cout << "Dime tu nombre." << std::endl;
    //program will block and wont follow this line until enter hit
    std::cin >> nombre;
    std::cout << "Hola, " << nombre << "!" << std::endl;
    return EXIT_SUCCESS;
}
\end{lstlisting}
\subsection{Subfiguras}
A veces, cuando se quieren mostrar varias imágenes relacionadas, se puede tener
la necesidad de crear figuras de figuras, es decir, una figura que se compone
de varias imágenes, cada una de las cuales tiene su propio título, en este
documento esto se puede ver en la Figura \ref{fig:compMarg}. El paquete que 
hay que incluir es \texttt{subcaption}. Para utilizar las subfiguras simplemente
se crea un entorno \texttt{subfigure} dentro del entorno figure. Aunque
hay que tener cuidado con elementos como la anchura de las figuras y subfiguras.
Un ejemplo de este uso sería esta figura compuesta de 3 imágenes:
\begin{figure}[H]
\centering
\begin{subfigure}{.3333\hsize}
    \centering
    \includegraphics[width=.9\linewidth]{naranja}
    \caption{Una naranja}
\end{subfigure}%
\begin{subfigure}{.3333\hsize}
    \centering
    \includegraphics[width=.9\linewidth]{manzana}
    \caption{Una manzana}
\end{subfigure}%
\begin{subfigure}{.3333\hsize}
    \centering
    \includegraphics[width=.9\linewidth]{pera}
    \caption{Una pera}
\end{subfigure}
\label{fig:frutas}
\caption{Unas cuantas frutas}
\end{figure}

Se ha incluido en las imágenes un marco para que se vean sus límites exactos.
El código utilizado para generar la figura compuesta ha sido este:
\begin{lstlisting}[language={[LaTeX]TeX},
                   caption={Ejemplo de subfigura},
                   label={lst:subfigure}]
\begin{figure}[H]
\centering
\begin{subfigure}{.3333\hsize}
    \centering
    \includegraphics[width=.9\linewidth]{naranja}
    \caption{Una naranja}
\end{subfigure}%
\begin{subfigure}{.3333\hsize}
    \centering
    \includegraphics[width=.9\linewidth]{manzana}
    \caption{Una manzana}
\end{subfigure}%
\begin{subfigure}{.3333\hsize}
    \centering
    \includegraphics[width=.9\linewidth]{pera}
    \caption{Una pera}
\end{subfigure}
\label{fig:frutas}
\caption{Unas cuantas frutas}
\end{figure}
\end{lstlisting}

El código que permite la creación de figuras compuestas es bastante explicativo,
pero hay ciertas salvedades que hay que comentar:
\begin{itemize}
\item Cada subfigura ocupa un tamaño concreto, por lo que hay que asegurarse
de que caben en la pantalla, en las líneas 4, 9 y 14 se puede ver que se ha 
configurado para que cada subfigura ocupe aproximadamente un tercio del espacio
horizontal disponible.
\item Además, la imagen que va dentro de cada subfigura no puede usar el 
parámetro \verb|\hsize|, sino \verb|\linewidth|. Ese parámetro mide tanto
como el ancho de la \textbf{subfigura}, por lo que es recomendable multiplicarlo
por $0,9$ para darle <<aire>> a las imágenes.
\item Entre el final de una subfigura y el inicio de la siguiente \textbf{debe
haber un comentario} para eliminar el salto de línea entre ellas, si no, las
subfiguras \textbf{no} saldrán a la misma altura, sino una encima de la
siguiente.
\item Cada subfigura puede tener su propia \textit{caption}, que se
numera con una letra. Se le puede añadir una etiqueta (\textit{label}), para
referencias pero en este caso no se ha hecho.
\end{itemize}

\subsection{Citas}
A veces queremos crear una cita en nuestro texto, por ejemplo en un artículo
científico, en vez de parafrasear la fuente, queremos incluir una cita de 
varias líneas. Quizás queremos una presentación más atractiva de una cita
célebre, o quizá queremos destacar una parte del texto, de un artículo,
por ejemplo. Para esto existe el entorno \textit{quote}, que nos permite
hacer un bloque de cita, por ejemplo:
\begin{quote}
«Un hippie es alguien que tiene las pintas de Tarzán, que camina como Jane y
que huele como Cheetah». 

-- Ronald W. Reagan, presidente de los Estados Unidos (1981-1989).
\end{quote}

El bloque anterior se ha creado simplemente inlcuyendo el texto en un entorno
\textit{quote}, con la salvedad de que se han creado dos párrafos, uno para
la cita y otro para indicar el autor.

Además, si se quieren decorar más, se pueden combinar con tamaños distintos, 
o colores, por ejemplo:
\begin{quote}
\LARGE\color{lightBlue}
\rule[2mm]{\linewidth}{.5pt}
<<Un hippie es alguien que tiene las pintas de Tarzán, que camina como Jane y
que huele como Cheetah>>. 
\rule[2mm]{\linewidth}{.5pt}
\normalsize

\color{blue}-- Ronald W. Reagan, presidente de los Estados Unidos (1981-1989).
\color{black}
\end{quote}

Se conseguiría así:
\begin{lstlisting}[language={[LaTeX]TeX},
caption={Ejemplo de citación con adornos}, label={lst:quoteWithDecoration}]
\begin{quote}
\LARGE\color{lightBlue}
\rule[2mm]{\linewidth}{.5pt}
<<Un hippie es alguien que tiene las pintas de Tarzan, que camina como Jane y
que huele como Cheetah>>. 
\rule[2mm]{\linewidth}{.5pt}
\normalsize

\color{blue}-- Ronald W. Reagan, presidente de los Estados Unidos. \color{black}
\end{quote}
\end{lstlisting}

En el ejemplo anterior se puede ver que simplemente se ha modificado el estilo
del texto con los comandos \verb|\LARGE| y \verb|\color|, se ha creado una
línea horizontal con el comando \verb|rule| y se ha escrito el texto, después,
se ha creado otra regla y se ha vuelvo a cambiar el estilo.
\subsection{Flotantes personalizados}
Además de los flotantes que hemos visto (\texttt{table},
 \texttt{figure} y \texttt{lstlisting}) se pueden definir flotantes propios
que nos permitan crear nuestras propias listas de rótulos. Por ejemplo, si
quisiéramos crear un flotante para incluir ecuaciones usaríamos el siguiente
código:
\begin{lstlisting}[language={
    [LaTeX]TeX},
    caption={Establecimiento de un flotante llamado ecuación},
    label={lst:newFloat}]
\usepackage{newfloat}
\usepackage{caption}
\DeclareFloatingEnvironment[fileext=ecc,
                            placement={H},
                            name=Ecuación,
                            listname={Índice de ecuaciones}]
                            {ecuacion}
\captionsetup[ecuacion]{labelfont=normal}
\end{lstlisting}

\begin{enumerate}
\item Incluye el paquete \emph{newfloat}.
\item Incluye el paquete \emph{caption}.
\item Empieza la orden y establece que la extensión del archivo para estos
flotantes es <<ecc>>, este parámetro debe ser \textbf{distinto} para cada
flotante que definamos (y distinto de los que ya existen).
\item Indica que estos flotantes se pondrán siempre donde sean escritos.
\item Indica el nombre que tendrán, el equivalente a <<tabla>> o <<figura>>.
\item Indica el nombre de la lista de este tipo de flotantes.
\item Indica el nombre de nuestro flotante (como lo pondremos en begin y end).
No puede tener ningún carácter especial, así que en este caso hemos escrito
<<ecuacion>> por <<ecuación>>.
\item Indica que el \emph{caption} de este flotante estará en estilo normal.
\end{enumerate}

Si quisiéramos incluir una ecuación con nombre el código que usaríamos sería:
\begin{lstlisting}[language={
    [LaTeX]TeX},
    caption={Inclusión de un flotante personalizado ecuación},
    label={lst:newFloat}]
\begin{ecuacion}
$$
G\cdot\frac{m_1\cdot m_2}{r^2}
$$
\caption{Ley de la gravedad}
\label{ecc:gravity}
\end{ecuacion}
\end{lstlisting}
El resultado sería:

\begin{ecuacion}
$$
G\cdot\frac{m_1\cdot m_2}{r^2}
$$
\caption{Ley de la gravedad}
\label{ecc:gravity}
\end{ecuacion}

\begin{enumerate}
\item Inicia el \emph{environment}.
\item Inicia el bloque matemático.
\item La fórmula en sí misma.
\item Termina el bloque matemático.
\item Incluye la \emph{caption}
\item Crea la \emph{label} para referencias. Por ejemplo, voy a referenciarla
aquí: Ecuación \ref{ecc:gravity}: \nameref{ecc:gravity}
\end{enumerate}

Para introducir la lista de nuestro nuevo flotante, en este caso, la lista de
ecuaciones, usaremos la orden \verb"\listof<nombre del flotante>s", por ejemplo,
en nuestro caso: \verb"\listofecuacions". \textbf{Cuidado con la s del final
que es necesaria}.

\newpage \section{Decoración e indización}
De momento no se han visto órdenes para crear índices, tablas de contenido,
pies de página, encabezados ni portadas como las que tiene este documento.
En esta sección veremos cómo se ha formateado este documento,
pero lo recomendable es que se quiere copiar alguna característica, simplemente
se incluya el código en tus propios documentos.
\subsection{La portada}
La clase article se creó para artículos científicos, así que no está preparada
para crear portadas tal y como las crean otros editores de texto como 
Word. Lo cual no es un gran inconveniente. Para crear la portada de este
documento se han seguido estos pasos:
\begin{enumerate}
    \item En el preámbulo del documento:
    \begin{enumerate}
        \item Se ha usado el comando \verb|\title|
        para indicar qué título ha de salir en la portada.
        \item Se ha usado el comando \verb|\author|
         para decir qué autor.
    \end{enumerate}
    \item En el contenido del documento:
    \begin{enumerate}
        \item Se ha usado el comando \verb|\maketile|
        para crear la portada propiamente dicha.
    \end{enumerate}
\end{enumerate}

En términos mínimos, en el documento .tex habrá ahora lo siguiente:
\begin{verbatim}
\documentclass{article}
\title{Título}
\author{Autor de Ejemplo}
\begin{document}
\maketitle
Primer texto del documento.
\end{document}
\end{verbatim}

Esto generará un documento con un título, el nombre del autor y la fecha,
pero el primer texto del documento saldrá en la misma página, y esa página
estará numerada, como 1. Además, la fecha estará en inglés. No es lo ideal.
Las siguientes tareas que llevaremos a cabo son:
\begin{enumerate}
\item Que la portada no se numere.
\item Que la portada esté en su propia página.
\item Modificar la fecha para que salga en español.
\end{enumerate}

Para conseguir la primera tarea, usaremos el comando 
\verb|\pagenumbering{gobble}|, que hace que las páginas
no se numeren. Este efecto dura hasta que se use otra orden que sobreescriba
este comportamiento.
Para la segunda, vamos a utilizar
el comando \texttt{newpage}, que hace que el siguiente texto salga en la página
siguiente. Ahora, para que las siguiente páginas sí se numeren, usaremos la
orden \verb|\pagenumbering{arabic}|,
así se numerarán con números arábigos.
Finalmente, para la tercera, vamos a utilizar el paquete babel, este
paquete <<traduce>> los textos automáticos (títulos, fechas...) a español,
y da mejor soporte a símbolos del español. En el preámbulo del documento
pondremos simplemente: \verb|\usepackage[spanish]{babel}|.
El contenido del documento debería ser, por tanto:
\begin{figure}[H]
\begin{verbatim}
\documentclass{article}

\title{Título}
\author{Autor de Ejemplo}

\usepackage[spanish]{babel}

\begin{document}
\maketitle
\pagenumbering{gobble}
\newpage
\pagenumbering{arabic}

Primer texto del documento.

\end{document}
\end{verbatim}
\end{figure}
Como colofón final, se puede hacer el texto del título y el autor más grande
utilizando los modificadores de tamaño del texto
(ver Tabla \ref{tab:comLetSize}: \nameref{tab:comLetSize}).

\subsection{Tablas de contenido}
Para insertar una tabla de contenido, simplemente se puede incluir en el 
contenido del documento, después de la portada (y de la orden
\texttt{newpage}, la orden
\verb|\tableofcontents|, si se desea que el siguiente
texto después de la tabla aparezca en otra página, se puede volver a usar
\texttt{newpage}. Recuerda incluir alguna sección para que el índice no esté
vacío, por ejemplo, incluiremos la sección <<prueba>>. Con estos cambios,
el documento quedaría así:

\begin{figure}[H]
\begin{verbatim}
\documentclass{article}
\title{Título}
\author{Autor de Ejemplo}
\usepackage[spanish]{babel}
\begin{document}
\maketitle
\pagenumbering{gobble}
\newpage
\pagenumbering{arabic}
\tableofcontents
\newpage
\section{Prueba}
Primer texto del documento.
\end{document}
\end{verbatim}
\end{figure}

Para los índices de figuras y tablas, existen las órdenes \texttt{listoffigures}
y \texttt{listoftables}. Recuerda añadir la orden \texttt{newpage} donde
quieras que el texto salga en la página siguiente. El único problema es que,
predeterminadamente, el paquete babel traduce <<tabla>> como <<cuadro>>, de tal
modo que el título de nuestro índice de tablas es <<Índice de cuadros>>, para
arreglar eso se debe añadir una opción al paquete babel, dejando la línea 
donde se incluye como:
\begin{verbatim}
\usepackage[spanish,es-tabla]{babel}
\end{verbatim}
\subsection{Encabezados y pies de página}
En algunas organizaciones o estilos, se recomienda que el índice -o índices-,
prólogos, prefacios y demás contenido anterior al \textbf{cuerpo} del texto
estén en páginas numeradas en números romanos y el resto en números arábigos.
Vale la pena mencionar que, en español, se desaconseja utilizar números romanos
en minúscula (\cite{raeRoman}). El resto de páginas (las del contenido del
texto), se numeran en números arábigos. Además, personalmente recomiendo, al
menos en el cuerpo del texto, incluir el número total de páginas. Esto es una
medida de integridad, así, en caso de transmisión o impresión del texto, si
alguna página del final se extraviare, el lector se daría cuenta de que faltan.

En general, hay que modificar los encabezados y pies de página. Para esto,
utilizaremos el paquete \texttt{fancyhdr}, que nos permite editar el contenido
en tres posiciones del encabezado y del pie de página: izquierda, central y 
derecha.

Para hacer esto, se deben poner las siguientes órdenes, estas órdenes pueden
ir en el preámbulo del documento o en el contenido. Si se ponen en el contenido
debe tener en cuenta que sólo afectarán a las páginas posteriores a su situación
en él, es una buena manera de definir numeraciones distintas en distintas
partes del documento como, de hecho, se ha hecho en este.

\begin{enumerate}
\item \verb|\pagestyle{fancy}|: esto indica que queremos
usar el estilo de encabezado y pie personalizado.
\item \verb|\fancyhf{}|: Dibuja una línea fina en el 
encabezado que enmarca la página.
\item \verb|\fancyhead[<pos>]{<contenido>}|: 
    \begin{enumerate}
        \item \texttt{<pos>} es la posición donde poner el contenido, se pueden
        poner: L, C, R, correspondientes a la izquierda, el centro y
        la derecha del encabezado.
        \item \texttt{<contenido>} es el contenido que se imprimirá en esa
        posición del encabezado. Puede ser texto o un comando que produzca
        contenido.
    \end{enumerate}
\end{enumerate}
Por ejemplo, las líneas:
\begin{verbatim}
\fancyhead[C]{Centro}
\fancyhead[L]{Izquierda}
\fancyhead[R]{Derecha}
\end{verbatim}
Generan el encabezado que se puede ver en la Figura \ref{fig:3columnHeader}: 
\nameref{fig:3columnHeader}.
\begin{figure}[H]
    \centering
    \includegraphics[width=\hsize]{columnasEncabezado}
    \caption{Ejemplo de encabezado a 3 columnas}
    \label{fig:3columnHeader}
\end{figure}

Existe, además, la orden
\verb|\fancyfoot[<pos>]{<contenido>}|, que es la que se usa
para añadir contenido al pie de página -en contraposición al encabezado, como 
cabamos de ver-. Voy a listar, también, algunos comandos útiles
para crear encabezados y pies más dinámicos:
\begin{enumerate}
\item \texttt{leftmark}: inserta el número y título del nivel de título más
alto que este justo antes de donde se inserte, por ejemplo, aquí insertaría:
<<\leftmark>>
\item \texttt{rightmark}: Añade el número y título del siguiente epígrafe
de nivel de título más grande, por ejemplo aquí insertaría <<\rightmark>>.
\item \texttt{thepage}: Añade el número de la página actual. Esto es muy
versátil porque añadirá el número de la página tal y como se haya numerado,
por ejemplo, este comando imprimirá números romanos si se configuró la 
numeración de la página como roman.
\item \texttt{pageref}: En general, pageref actúa como \texttt{ref} pero inserta
el número de página donde está la etiqueta, esto permite por ejemplo indicar
el número de páginas de una sección si se pone una etiqueta justo al final de la
misma.
\item \texttt{LastPage}: Requiere el paquete lastpage, es una etiqueta 
automática al final del texto, así, se puede hacer
\verb|\pageref{LastPage}|
\item \texttt{today}: inserta la fecha de hoy.
\end{enumerate}

Uno puede ser muy creativo con esta parte del diseño de su documento, por 
ejemplo, aquí se incluye el código que se ha usado para esta sección del
documento:
\begin{lstlisting}[language={[LaTeX]Tex}, caption={Establecimiento del
encabezado y pie de página},
label={lst:definecolor}]
\pagestyle{fancy}
\fancyhf{}
\fancyhead[C]{
    \Large\titulo\normalsize\\
    \rule[1mm]{0.3\hsize}{.5pt}\\
    \leftmark
}
\fancyhead[R]{\includegraphics[height=2cm]{escudo}}
\fancyfoot[R]{pág. \thepage{} de \pageref{LastPage}}
\fancyfoot[L]{\today}
\renewcommand{\footrulewidth}{0.5pt}
\end{lstlisting}
\begin{enumerate}
\item Habilitamos estilo a 3 columnas.
\item Creamos la línea horizontal del encabezado
\item Establece que el texto que se indique va a ir en el centro
del encabezado.
\item Imprime el título a tamaño grande e inserta un salto de línea.
\item Inserta una línea de un tercio del ancho de la página, de grosor
de medio punto y elevada 1 mm sobre la línea base de su párrafo.
\item Inserta la sección actual.
\item Cierra el comando de la línea 3.
\item A la derecha incluímos una imagen con un alto de 2 cm. Aconsejo fijar el
alto -en lugar del ancho- y usar imágenes más o menos cuadradas.
\item Inserta el texto <<Pág. <XX> de <totales>{}>>. En la parte derecha del
\textbf{pie} de página.
\item Inserta la fecha actual en la parte izquierda del pie de página.
\item Redefine el grosor de la línea del pie de página que es, 
predeterminadamente, invisible.
\end{enumerate}

Además, si en la clase del documento se activa la opción \texttt{twoside}
cambiando la línea donde se indica por esta:
\verb|\documentclass[twoside]{article}| podremos seleccionar
no sólo la columna donde aparece el contenido, sino también si lo hace en
páginas pares o impares del documento. Por ejemplo:
\begin{verbatim}
\fancyhead[RE,LO]{\Huge\LaTeX\normalsize}
\end{verbatim}
haría que el texto <<\LaTeX>> apareciera en el encabezado en la derecha
en las páginas pares (E es por \emph{even} y O por \emph{odd}, par e impar
respectivamente en inglés) y en la izquierda en las impares. El resultado se
puede ver en la Figura: \ref{fig:twoside}.
\begin{figure}[H]
\centering
\includegraphics[width=\hsize]{twoside}
\caption{Ejemplo de documento para impresión a doble cara}
\label{fig:twoside}
\end{figure}
Hay que tener cuidado, porque si una página es par o impar viene dado por su
\textbf{número de página}, no por su posición en el documento de manera
absoluta, esto quiere decir que una página que se considera par por estar 
numerada como dos puede imprimirse en una cara impar si anteriormente hay un
número impar de páginas. Para forzar a que el siguiente texto empiece en página
par se usaría el comando \verb|\cleardoublepage|. Si se va
a imprimir el documento a dos caras, sería recomendable usarlo tras la portada
(para dejar una página en blanco que sería su parte de atrás) y después de los
índices.
\subsection{Tamaño de página y márgenes}
Las clases de documento de \LaTeX{} tienen una distribución de página 
predeterminada, en el caso de la clase \emph{article}, éstos son unos márgenes
bastante anchos, a continuación se puede ver una comparación entra 
configuración normal y la configuración que se ha hecho para este documento.
\begin{figure}[H]
\centering
    \begin{subfigure}{.5\hsize}
        \centering
        \includegraphics[width=0.9\linewidth]{margenesAnchos}
        \caption{Márgenes Anchos (predeterminados)}
        \label{fig:marg1}
    \end{subfigure}% <- este paréntesis es super importantísimo
    \begin{subfigure}{.5\hsize}
        \centering
        \includegraphics[width=0.9\linewidth]{margenesEstrechos}
        \caption{Márgenes estrechos (modificados)}
        \label{fig:marg2}
    \end{subfigure}
    \caption{Comparación de márgenes}
    \label{fig:compMarg}
\end{figure}
Las órdenes que se han usado son las siguientes:
\begin{verbatim}
1   \usepackage{geometry}
2   \geometry{
3       a4paper,
4       left=30mm,
5       bottom=35mm,
6       headheight=25mm
7   }
\end{verbatim}
\begin{enumerate}
\item Inclusión del paquete \emph{geometry}, que permite la configuración.
\item Comando necesario para definir la geometría.
\item Define el tamaño del paper (DIN-A4). El tamaño predeterminado de la clase
\textit{article} no es DIN-A4, sino el tamaño carta del ANSI. En caso de no
querer redefinir más opciones de la geometría, pero desear el tamaño DIN-A4,
se puede indicar en la clase del documento con esta opción:
\verb|\documentclass[a4paper]{article}|.
\item Margen lateral.
\item Margen inferior.
\item Tamaño del encabezado.
\end{enumerate}
\subsection{Coloreado} \label{coloreado}
LaTeX ofrece la posibilidad de colorear el texto \textit{ad-hoc} si así lo
deseamos, por ejemplo \\\verb|\color{red} Texto\color{black}| generaría:
<<\color{red} Texto\color{black}>>. Pero el catálogo de colores es limitado,
por lo que se puede incluir el paquete \texttt{xcolor}, que incluye
la capacidad de definir nuevos colores con el comando \verb|definecolor|, 
que se usa de este modo:
\begin{lstlisting}[language={[LaTeX]Tex}, caption={Uso de \texttt{definecolor}},
label={lst:definecolor}]
%Esquema de uso
%\definecolor{<nombre del color>}{rgb}{rValue, gValue, bValue}
%Ejemplo real
\definecolor{gray}{rgb}{0.5,0.5,0.5}
\end{lstlisting}
El primer argumento del comando es el nombre que le vamos a dar al color,
el segundo es el esquema de colores que vamos a usar, existen varios:
\begin{enumerate}
\item rgb: recibe tres cifras representando los valores de rojo, verde y azul
del color, normalizadas entre cero y uno.
\item RGB: como el anterior, pero las cifras están entre cero y 255.
\item hsb: Recibe valores de tono, saturación y luminosidad entre cero y uno.
\item HSB: Igual que el anterior, pero los valores van de 0 a 240.
\end{enumerate}
Y, finalmente, el último argumento son los tres valores que necesite nuestro
esquema de color.
\subsection{Texto a dos columnas}

Si se desea introducir texto a dos columnas, debe incluirse el paquete
\verb!multicol!, y utilizar el entorno \verb!multicols!, que nos
permite indicar el número de columnas del texto. 

\begin{lstlisting}[ language={[LaTeX]Tex},
                    caption={Ejemplo de texto a dos columnas},
                    label={lst:two_columns}]
\begin{multicol}{2}
%el texto que se desee que salga a dos columnas
\end{multicol}
\end{lstlisting}

En la figura \ref{fig:two_columns} se puede ver el resultado.

\begin{figure}[H]
\includegraphics[width=\hsize]{two_columns}
\caption{Texto a dos columnas}
\label{fig:two_columns}
\end{figure}

\newpage\section{Presentaciones de transparencias}
Como se dijo en \textbf{\ref{sec:presentacion}}, se puede utilizar \LaTeX{} para
crear presentaciones de transparencias, usando una clase de documento
muy especial llamada \textit{beamer}. Ya se ha explica que  es muy difícil, si
no imposible, cambiar la clase de un documento cuando éste tiene cierta
extensión. Por lo que se ha creado otro documento llamado \texttt{beamer.tex}
que se compila con la orden \verb!make beamer.pdf! si se dispone de soporte para
compilar con \verb!make!. En este documento incluiré el código para crear la
presentación de transparencias del mismo modo que he hecho con el que ya 
se ha visto, pero incluiré, además, capturas del resultado.
 
La estructura básica de un documento con la clase beamer es la misma que la de 
uno con cualquier otra clase de documento, pero el contenido
del mismo es lo que cambia:
\begin{lstlisting}[language={[LaTeX]TeX},
                   label={lst:basicBeamer},
                   caption={Documento básico con \textit{beamer}}]
\documentclass{beamer}
\begin{document}
    \begin{frame}
        Aquí ya puede ir algún texto, para que se incluya en la primera
        transparencia.
    \end{frame}
\end{document}
\end{lstlisting}
Lo primero que llama la atención es el uso del entorno \verb!frame!, que es
el que nos marca una transparencia. 
Así, como se puede ver en, la figura \ref{fig:slide_text},
se ha creado una transparencia
con texto en el medio. Además, la clase \textit{beamer} incluye unos controles
de nevegación en la parte inferior derecha de todas las transparencias sin que
tengamos que hacer nada.
\begin{figure}[H]
    \includegraphics[width=\hsize]{slide_text}
    \caption{Una transparencia simple con texto}
    \label{fig:slide_text}
\end{figure}

Hay que tener cuidado, porque \LaTeX{} no maneja el tamaño del texto de la
transparencia por lo que, si introducimos mucho, éste se desbordará como se
puede ver en la figura \ref{fig:slide_overflow}

\begin{figure}[H]
\includegraphics[width=\hsize]{slide_overflow}
\caption{Una transparencia en la que el texto se desborda}
\label{fig:slide_overflow}
\end{figure}

Si incluimos varios entornos \textit{frame}, crearemos varias transparencias.

\begin{lstlisting}[language={[LaTeX]TeX},
                   label={lst:slide_two},
                   caption={Código para dos transparencias}]
\documentclass{beamer}
\begin{document}
    \begin{frame}
        Aquí ya puede ir algún texto, para que se incluya en la primera
        transparencia.
    \end{frame}
    \begin{frame}
        Aquí iría el texto de la siguiente transparencia.
    \end{frame}
\end{document}
\end{lstlisting}

En la figura \ref{fig:slide_two} se pueden ver dos transparencias.
\begin{figure}[H]
\includegraphics[width=\hsize]{slide_two}
\caption{Dos transparencias}
\label{fig:slide_two}
\end{figure}

Antes de continuar, hay que hacer una serie de configuraciones y explicar una
serie de salvedades sobre la clase \textit{beamer}:
\begin{enumerate}
    \item Beamer incluye en paquete \verb!amsmath! automáticamente, así que
    debemos tener en cuenta lo que se explica en \textbf{\ref{sec:amsmath}}.
    \item La fuente predeterminada de los documentos \verb!beamer! es sin
    serifas, incluso en ecuaciones matemáticas, si se quiere utilizar la fuente
    predeterminada del resto de documentos, debe configurarse usando la
    orden \verb!\usefonttheme{serif}! en el preámbulo.
    \item Del mismo modo que pasaba con las tablas en este documento,
    para que los textos automáticos
    salgan en español, debe configurarse el idioma mediante el paquete
    \verb!babel!, pero requerimos, además, de dos instrucciones adicionales
    \begin{lstlisting}[language={[LaTeX]TeX},
                       label={lst:babel_beamer},
                       caption={Configuración de \texttt{beamer} en español}]
\usepackage[spanish]{babel}
\uselanguage{Spanish}
\languagepath{Spanish}
    \end{lstlisting}
\end{enumerate}

Una de las funciones básicas de las transparencias es ponerle un título a las
mismas, para eso hay que utilizar la orden \verb!\frametitle{<título>}!, de
este modo, el título aparecerá en la parte superior de la transparencia.
Además, si queremos que nuestras presentaciones sean más llamativas 
debemos usar un <<tema>>, la manera de establecer este tema es con la orden
\verb!\usetheme{<nombre del tema>}!. Por ejemplo, un de los temas más usados
es el tema Madrid, así que en el preámbulo podemos introducir esa orden. Así,
las trasnparencias quedarían como se ven en la figura \ref{fig:slide_madrid}.
\begin{figure}[H]
\includegraphics[width=\hsize]{slide_madrid}
\caption{Transparencias con el tema Madrid}
\label{fig:slide_madrid}
\end{figure}
El código para crearlas es el siguiente:

\begin{minipage}[H]{\hsize}
\begin{lstlisting}[language={[LaTeX]TeX},
                   label={lst:slide_madrid},
                   caption={Código de beamer usando un tema y títulos}]
\documentclass{beamer}
\usetheme{Madrid}
\begin{document}
    \begin{frame}
        \frametitle{Primera transparencia}
        Aquí ya puede ir algún texto, para que se incluya en la primera
        transparencia.
    \end{frame}
    \begin{frame}
        \frametitle{Segunda transparencia}
        Aquí iría el texto de la siguiente transparencia.
    \end{frame}
\end{document}
\end{lstlisting}
\end{minipage}

Como se puede ver, en la línea 2 se incluye el tema y en las líneas 5 y 10 se
indican los títulos de las transparencias.

Ahora que hemos visto lo básico de las transparencias, vamos a ver cómo crear
una transparencia de título, una tabla de contenidos, bloques destacados y 
transparencias a dos columnas.
Para crear una transparencia de título se hace de un modo parecido a como
se hace en un documento normal, se deben usar las órdenes \verb!\title!,
\verb!\author! y, esta sí cambia: \verb!\titlepage!. Para crear una tabla de
contenido, se debe utilizar la orden \verb!\tableofcontents! dentro de un
\textit{frame}. Veamos un ejemplo completo de cómo incluir autores, títulos,
subtítulos e instituciones:

\begin{minipage}[c]{\hsize}
\begin{lstlisting}[language={[LaTeX]TeX},
                   label={lst:slide_titles},
                   caption={Información de título en \texttt{beamer}}]
\author[Rodríguez y Brown]{
    Francisco Rodríguez Melgar\inst{1}\\
    Dr. Emmett Lathrop Brown\inst{2}\\
}
\title[Ejemplo beamer]{
    Ejemplo de presentación con \LaTeX{}
}
\subtitle{
    Cómo hacer presentaciones con \LaTeX{} y Beamer
}
\institute[UdE y Caltech]{
    \inst{1}Universidad de Ejemplo \\ Facultad de las cosas que molan\and
    \inst{2}Instituto Tecnológico de California\\Facultad de Viajes en el Tiempo

}
\date[Esp. Conf.]{
    Conferencias Espaciales -- \today
}
\end{lstlisting}
\end{minipage}

En la línea 1 se especifica el autor con el comando correspondiente, el
argumento entre corchetes es la versión corta de los autores, que se mostrará
en la barra inferior de las transparencias, después, entre llaves, en las líneas
2 y 3 se especifica cada autor separado por un salto de línea (\verb!\\!).
Los comandos
\verb!\inst{<número>}! permiten colocar un superíndice con un número que nos
permitirá relacionar estos autores con las instituciones a las que pertencen.
En la línea 5 se indica del mismo modo la versión corta del título de la
presentación, y en la 6, la versión larga. En la 8, 9 y 10 se indica el
subtítulo. En la 11 se especifican las instituciones con el comando
\verb!\institute!, que nos permite, con el comando \verb!\and!, indicar más
de una. Como se puede ver, se han utilizado los comandos \verb!\inst! antes
de cada institución para asignales ese número y que aparezca en la
transparencia. Finalmente, en las líneas 16 a 18 se indica el marco de
conferencias en que se enmarca la presentación, del mismo modo, entre corchetes
la versión corta (que aparecerá en la barra inferior) y entre llaves en versión
larga, que aparecerá en la transparencia de título. Nótese que se debe usar
el comando \verb!\today! para imprimir la fecha de creacion del documento,
si no, sólo imprimiría el nombre de la conferencia.
El resultado se puede ver en la figura \ref{fig:slide_title}.

\begin{figure}[H]
\includegraphics[width=\hsize]{slide_title}
\caption{Transparencia de título}
\label{fig:slide_title}
\end{figure}

Hay que tener en cuenta que los títulos de las transparencias que se introducen
con la orden \verb!\frametitle! \textbf{no son} los que aparecen en la tabla de
contenido, los que aparecen son los títulos de secciones y subsecciones, que se
usan del mismo modo que en documentos normales. Nótese que estas órdenes deben
ir fuera del entorno frame. En la figura \ref{fig:slide_content} se puede ver
cómo quedaría una tabla de contenido con siete secciones y una subsección.

\begin{figure}[H]
\includegraphics[width=\hsize]{slide_content}
\caption{Transparencia de tabla de contenido}
\label{fig:slide_content}
\end{figure}

En algunas transparencias, quizás, queramos destacar un párrafo sobre los
demás, porque, quizá, sea un teorema importante, para ello existen las órdenes:
\begin{itemize}
\item \textbf{block}: Nos permite introducir un bloque del color del tema con
un título.
\item \textbf{alertblock}: Nos permite introducir un bloque de color destacado
(rojo, por ejemplo).
\item \textbf{examples}: Crea un bloque de color intermedio que tiene de título
<<Ejemplos>>. (Esto es posible gracias a la configuración de babel).
\end{itemize}

En el programa \ref{lst:alertblocks} se puede ver un 
ejemplo de cómo se introducen;
y en la figura \ref{fig:slide_alertblocks} se puede ver el resultado.

\begin{minipage}[c]{\hsize}
\begin{lstlisting}[language={[LaTeX]TeX},
                   label={lst:alertblocks},
                   caption={Ejemplo de bloques alerta de beamer}]
% resto de transparencias, configuración...
\begin{frame}
    \frametitle{Introducción}
        Aquí iría el texto de la siguiente transparencia.
    \begin{block}{Teorema importante}
    \[
        |x|=\left\{
        \begin{matrix}
            -x&\textrm{para}&x<0\\
            x&\textrm{para}&x\geq0
        \end{matrix}
        \right.
    \]
    \end{block}
    \begin{alertblock}{Entrega del trabajo}
        Recordad que el trabajo sobre \textit{cosas importantes} se debe
        entregar el lunes que viene.
    \end{alertblock}
    \begin{examples}
        \[
            |-2|=2;\:|2|=2
        \]
    \end{examples}
\end{frame}
% más cosas...
\end{lstlisting}
\end{minipage}

En las líneas 5, 15 y 19 se puede ver la sintaxis para iniciar el entorno
que nos introduce los bloques. Además, entre las líneas 6 y 13 podemos ver cómo
se ha utilizado la notación matemática compatible con el paquete \verb!amsmath!.
\begin{figure}[H]
\includegraphics[width=\hsize]{slide_alertblocks}
\caption{Ejemplo de bloques}
\label{fig:slide_alertblocks}
\end{figure}

Por otro lado, las transparencias estilizan las listas numeradas o sin numerar
de una manera atractiva dependiente del tema que se elija. Por ejemplo:
\begin{figure}[H]
    \begin{subfigure}{.5\hsize}
        \centering
        \includegraphics[width=.9\linewidth]{slide_enumerate}
        \caption{Lista numerada en transparencia}
    \end{subfigure}%
    \begin{subfigure}{.5\hsize}
        \centering
        \includegraphics[width=.9\linewidth]{slide_itemize}
        \caption{Lista no numerada en transparencia}
    \end{subfigure}
    \caption{Ejemplo de listas en transparencias}
    \label{fig:slide_lists}
\end{figure}

Para crear transparencias a dos columnas se hace con el entorno \verb!columns!.
Dentro de él, se introducen tantos entornos \verb!column! como se deseen, que
serían las columnas de la transparencia. Para cada columna debe decirse el
ancho, en un argumento el entorno \verb!column!. Veamos un ejemplo:
\begin{minipage}[c]{\hsize}
\begin{lstlisting}[language={[LaTeX]TeX},
                   label={lst:slide_two_columns},
                   caption={Ejemplo de transparencia a dos columnas}]
\begin{frame}
    \frametitle{Transparencia}
    \begin{columns}
        \begin{column}{0.5\hsize}
            Lorem ipsum dolor sit amet, conse
            ctetur adipiscing elit. 
            Integer eleifend lectus eros,
            at dignissim leo puortor et, auctor arcu. Ut odio 
            quam, faucibus non enim quis, consectetur rhoncus nisl.
        \end{column}
        \begin{column}{.5\hsize}
            \begin{itemize}
                \item Alemania
                \item España
                \item Andorra
                \item Noruega
            \end{itemize}
        \end{column}
    \end{columns}
\end{frame}
\end{lstlisting}
\end{minipage}

En la línea 3 se empieza el entorno \verb!columns!, en las líneas 4 y 11 se 
empiezan los respectivos entornos \verb!column!. Nótese que después de 
\verb!{column}!, se debe poner el tamaño de la columna, pero que \textbf{no} 
se debe poner \verb!width=!, si se pusiera, fallaría. Después, simplemente
se introduce el texto (u objetos) que se deseen en los entornos. En la figura
\ref{fig:slide_two_columns} se puede ver el resultado.
\begin{figure}[H]
\includegraphics[width=\hsize]{slide_two_columns}
\caption{Transparencia a dos columnas}
\label{fig:slide_two_columns}
\end{figure}

Como detalle, el entorno \verb!beamer! no permite entornos verbatim dentro de
él, por lo que si se desea usar ese o el entorno \verb!lstlisting!, debe
añadirse a la transparencia la opción \verb![fragile]!, es decir, la línea
quedaría como: \verb!\begin{frame}[fragile]!. Aparte de eso, el paquete
\verb!xcolor! debe incluirse sin opciones, porque la clase de documento
\verb!beamer! ya incluye algunas. Por otro lado, debe incluirse una opción
llamada \verb!xleftmargin! en el bloque \verb!lstset! con un valor de, por
ejemplo, \verb!2em!, o si no los números de línea (por la izquierda), saldrían
fuera de la transparencia. El resultado de un código introducido como
lstlisting en una transparencia se puede ver en la figura \ref{fig:slide_code}.
\begin{figure}[H]
\includegraphics[width=\hsize]{slide_code}
\caption{Código en una transparencia}
\label{fig:slide_code}
\end{figure}

En general, todas las reglas que aplican para documentos normales
aplicarían para la clase de documento \verb!beamer!, por lo que simplemente
sería ir probando lo que necesites. Por ejemplo, las figuras se insertan
del mismo modo que en documentos de otras clases, etc.

\newpage\section{Licencias y agradecimiento}
Todas las páginas citadas son de sus respectivos creadores, yo sólo me limito
a recopilar sus soluciones o información en un mismo documento. Además, quiero
recoger aquí los nombres de los creadores de los paquetes utilizados y darles
las gracias por permitir que la comunidad haga documentos con funcionalidades
avanzadas.

\begin{table}[H]
\centering
\begin{tabular}{|c|c|}
\hline
\textbf{Paquete}&\textbf{Autor}\\\hline
fontenc&El equipo \LaTeX{} \\\hline
titlesec&Javier Bezos\\\hline
babel&Johannes Braams, Javier Bezos \textit{et al.} \\\hline
eurosym& Henrik Theiling\\\hline
graphicx&David Carlisle, Sebastian Rahtz\\\hline
float& Anselm Lingnau \\\hline
hyperref&Sebastian Rahtz, Heiko Oberdiek, El equipo \LaTeX{}\\\hline
fancyhdr& Pieter van Oostrum\\\hline
lastpage&Hans-Martin Münch, Jeffrey Goldberg \\\hline
biblatex&Philipp Lehman, Joseph Wright, Audrey Boruvka y Philip Kime\\\hline
geometry&Hideo Umeki, David Carlisle\\\hline
inputenc&Frank Mittelbach, Alan Jeffrey y el equipo \LaTeX{} \\\hline
amsfonts&American Mathematical Society\\\hline
subcaption&Axel Sommerfeldt\\\hline
listings& Carsten Heinz, Brooks Moses y Jobst Hoffmann\\\hline
xcolor& Dr. Uwe Kern\\\hline
cancel&Donald Arseneau\\\hline
multirow&Jerry Leichter, Pieter van Oostrum \\\hline
array& El equipo \LaTeX{} \\\hline
rotating&Sebastian Rahtz, Leonor Barroca, Robin Fairbairns y El equipo \LaTeX{}  \\\hline
\end{tabular}
\caption{Paquetes utilizados y sus creadores}
\label{tab:packagesUsed}
\end{table}
Todas las
imágenes utilizadas son de licencia libre (ver:
\href{https://pixabay.com/service/faq/}{FAQ de Pixabay}) o de creación propia.
El nombre de Caltech se usa con fines humorísticos haciendo referencia al
personaje \textbf{ficticio} del cine Dr. Emmett Brown, ni el autor ni este 
documento tienen ninguna relación con dicha institución.
\newpage
\section{Bibliografía}
\printbibliography[heading=none]
\end{document}
